The rover SherpaTT has been deployed in several field experiments where it was put under test within natural and unstructured Mars analogue terrain with respect to general morphology and geology. The rover displayed the ability to cope with natural terrain and to fulffill the task of being an exploration and sampling rover. Currently, the rover is prepared for further autonomous long distance traverses in terrain akin to the Martian environment. However, it features a fueled power generator which cannot be employed in Martian or Lunar missions and limits the system's autonomous mission lifetime to the discharge rate of its two LiPo batteries.

As the rover is meant to approach a higher technology readiness level, further development is required on its electrical power subsystem if it is to operate in long term missions. This thesis will explore solar array configurations with respect to constraints imposed by the Martian environment so that future versions of the rover may be designed to navigate the topography of this planetary surface. The constraints imposed from the active suspension system with flexible footprints and varying heights of structural parts of the legs are considered in the design phase.

Mars mission scenarios will be explored in order to propose solar array requirements and impose power storage and consumption constraints based on sol-by-sol analysis of solar insolation as a function of geographic latitude, areocentric longitude, atmospheric opacity, dust deposition, and sun angle of incident on the solar array. The mission scenario will put particular emphasis on including topograpy best tackled by the rover's wheeled leg design, such as climbing up mountains and volcanoes or going down valleys and craters. Sites with high geological and exobiological science potential typically require navigating complicated terrains which offer an opportunity for further power budget analysis that take into account solar insolation on an inclined surface.

\todo[inline]{TODO: Reference DFKI literature regarding SherpaTT, specifically the field trials.}

\todo[inline]{TODO: Expand the introduction to introduce the structure of the thesis.}


%A Lunar mission scenario will also be investigated with particular interest on access to permanently shadowed areas in the southern pole. The mission scenarios will put particular emphasis on including topograpy best tackled by the rover's wheeled leg design, such as climbing up mountains and volcanoes or going down valleys and craters. Sites with high geological and exobiological science potential typically require navigating complicated terrains which offer an opportunity for further power budget analysis that take into account solar irradiance on slopes.
