Contents of this chapter were taken from \citeother{CDF2014} and constrained for the scope of this thesis. Adjustements were made in relation to the mission sites described in the previous chapter as well as with the rover's capabilities with respect to its active suspension system. Rover modes were identified and then sequenced into high-level Sol mission scenarios. These scenarios served as reference Sols for mission planning. Defining rover modes and reference Sols served as a prerequisite to the following chapter were preliminary power budgets were determined by associating each rover modes with their respective operational duration and power draw.

The chapter is structured as follows: Section \ref{sec:ReferenceSols:RoverModes} provides general definitions of rover modes. The identified rover modes are sequenced into reference Sols in Section \ref{sec:ReferenceSols:ReferenceSols}. The chapter is then summarized in Section \ref{sec:PowerBudget:SummaryAndConclusion}.

\section{Rover Modes}
\label{sec:ReferenceSols:RoverModes}
Not all modes defined in \citeother{CDF2014} were included in this section, however; they may become of interest in future design iterations. The omitted modes are \textit{Launch}, \textit{\ac{EDL}}, \textit{Deployment}, \textit{Science Stop - Long},  and \textit{Safe} modes. The rover modes considered in this thesis are as follow:

\begin{itemize}
    \item \textbf{Traverse - Flat:} Pre-planned flat surface traverses to target destinations.
    \item \textbf{Traverse - Upslope:} Pre-planned upslope surface traverses to target destinations.
    \item \textbf{Science Stop - Short:} Short science activities in between traverses.
    \item \textbf{\ac{DTE} Communication:} Pre-planned communication sessions.
    \item \textbf{Idle - Day:} All day dedicated to charging batteries.
    \item \textbf{Idle - Night:} Minimal battery usage during night.
    \item \textbf{Hibernation:} Survival mode during dust storms ($\tau \geq 2$).
    \item \textbf{Optimal Pose:} Repositioning the rover along its principal yaw, pitch, and roll axes to maximized next Sol solar array power generation.
\end{itemize}
%\ac{DTE} Communication operations are encapsulated as their own mode whereas \ac{UHF} Communications are mode activities conducted during \textit{Traverse}, \textit{Science Stop}, and \textit{Idle} modes.

%Communication operations and their schedules were also taken from \citeother{CDF2014}:
%\begin{itemize}
%    \item Uplink of sol plan either early in the morning (\SI{30}{\minute} \ac{DTE} or \SI{7}{\minute} \ac{UHF}) or during the night (\SI{7}{\minute} \ac{UHF}).
%    \item Downlink of high priority data required for planning of the next sol either in the afternoon (\SI{5}{\minute} \ac{DTE}) or during the night (\SI{7}{\minute} \ac{UHF}).
%\end{itemize}

The \textit{Optimal Pose} mode was not taken from \citeother{CDF2014}. It was created for the purpose of this study so that the rover may use its active suspension system as a solar tracking mechanism to maximize \ac{SA} power generation. For the sake of simplicity, a constraint was applied to the pitch and roll axes which mutually excluded them from both being actuated. Thus, during the \textit{Optimal Pose} mode, only yaw-pitch or yaw-roll rotation combinations are allowed. The inclination angle resulting from a pith or roll rotations is the $\beta$ angle. The orientation angle with respect to the direction of the equator which resulting from a yaw rotation is the surface azimuth angle $\gamma_{c}$. These angles were introduced in Section \ref{sec:MartianEnvironment:SolarRadiation:InclinedSurface}.

Rover subsystems as well as their states during the rover modes were taken from \citeother{CDF2014} and are shown in Table \ref{tab:rover-modes}.

 %Some mode durations were adjusted from those presented in \citeother{CDF2014} in response to power budgeting constraints pertaining to the scope of this research.

\begin{table}[h]
\small
\centering
\caption{Rover modes}
\label{tab:rover-modes}
\begin{tabular}{|l|l|l|l|}
\hline
\textbf{ID} & \textbf{Name} & \textbf{Description} & \textbf{\begin{tabular}[c]{@{}l@{}}Duration\\ per Sol\end{tabular}} \\ \hline
\textbf{RM-01} & Traverse & \begin{tabular}[c]{@{}l@{}}\textbf{Pre-planned traverses to target destinations.} \\ Daytime only. Batteries not charging.\\ Propulsion ON.\\ Navigation ON.\\ Communication ON.\\ Science OFF.\\ Heaters OFF.\end{tabular} & \begin{tabular}[c]{@{}l@{}}Up to\\ 180 min\end{tabular} \\ \hline
\textbf{RM-02} & \begin{tabular}[c]{@{}l@{}}Science Stop - Short\end{tabular} & \begin{tabular}[c]{@{}l@{}}\textbf{Short science activities in between traverses.} \\ Daytime only. Batteries charging.\\ Propulsion OFF.\\ Navigation OFF.\\ Communication ON.\\ Science ON.\\ Heaters OFF.\end{tabular} & 60 min \\ \hline
\textbf{RM-03} & \begin{tabular}[c]{@{}l@{}}Science Stop - Long\end{tabular} & \begin{tabular}[c]{@{}l@{}}\textbf{All day dedicated to science activities.} \\ Daytime only. Batteries charging.\\ Propulsion OFF.\\ Navigation OFF.\\ Communication ON.\\ Science ON.\\ Heaters OFF.\end{tabular} & 240 min \\ \hline
\textbf{RM-04} & \begin{tabular}[c]{@{}l@{}}Commnunication\\ \ac{DTE} \end{tabular} & \begin{tabular}[c]{@{}l@{}}\textbf{Planned communication sessions.} \\ Daytime only. Batteries not charging.\\ Propulsion OFF.\\ Navigation OFF.\\ Communication ON.\\ Science OFF.\\ Heaters OFF.\end{tabular} & 5-30 min \\ \hline
\textbf{RM-05} & Idle - Day & \begin{tabular}[c]{@{}l@{}}\textbf{All day dedicated to charging batteries.} \\ Daytime only. Batteries charging.\\ Propulsion OFF.\\ Navigation OFF.\\ Communication ON.\\ Science OFF.\\ Heaters OFF.\end{tabular} & All day \\ \hline
\textbf{RM-06} & Idle - Night & \begin{tabular}[c]{@{}l@{}}\textbf{Minimum battery usage during night.} \\ Night-time only. Batteries not charging.\\ Propulsion OFF.\\ Navigation OFF.\\ Communication ON.\\ Science OFF.\\ Heaters ON.\end{tabular} & All night \\ \hline
\textbf{RM-07} & Hibernation & \begin{tabular}[c]{@{}l@{}}\textbf{Survival mode during dust storms ($\tau \geq 1$).}\\ Daytime and/or night-time. Batteries not charging.\\ Propulsion OFF.\\ Navigation OFF.\\ Communication OFF.\\ Science OFF.\\ Heaters ON.\end{tabular} & \begin{tabular}[c]{@{}l@{}}All day\\ and night\end{tabular} \\ \hline
\end{tabular}
\end{table}




\section{Reference Sols}
\label{sec:ReferenceSols:ReferenceSols}
Flat and upslope traverses are the worst-case modes in terms of energy consumption at two different topographies. Hibernation must be constrained with respect to battery depletion in order to appreciate the rover's survivabilty during a dust storm. As such, only the \textit{Flat Traverse}, \textit{Upslope Traverse}, and \textit{Hibernation} reference Sols required analysis.


\subsection{Traverse Sol}
\label{sec:ReferenceSols:TraverseSol}
A \textit{Traverse Sol} allows the rover to reach a pre-planned target destination. Flat and upslope traverses have identical mode sequences, shown in Table \ref{tab:mission-scenario-traverse-sol}, only differing in the duration of their respective \textit{Traverse} modes. This distinction relates to propulsion power draw differences that are presented in Chapter \ref{sec:PowerBudget}.

\begin{table}[h]
\footnotesize
\centering
\caption{Reference Sol for flat or upslope traverse.}
\label{tab:mission-scenario-traverse-sol}
\begin{tabular}{|l|l|}
\hline
\textbf{Mode} & \textbf{Description} \\ \hline
\textbf{Idle - Day} & \begin{tabular}[c]{@{}l@{}}Pre-Heating.\\ UHF Communication.\\ Battery Charging.\end{tabular} \\ \hline
\textbf{DTE Communication} & \begin{tabular}[c]{@{}l@{}}Downlink.\\ Uplink.\end{tabular} \\ \hline
\textbf{Traverse - Flat or Upslope} & Propulsion to target destination. \\ \hline
\textbf{Science Stop - Short} & Science operations. \\ \hline
\textbf{Optimal Pose} & Maximize SA power generation for next Sol. \\ \hline
\textbf{Idle - Night} & \begin{tabular}[c]{@{}l@{}}UHF Communication.\\ Thermal Regulation.\end{tabular} \\ \hline
\end{tabular}
\end{table}


A Traverse Sol begins with the \textit{Idle - Day} mode, which is mostly dedicated to recharging the rover's batteries. On the previous Sol, the \ac{SA} was tilted and oriented into an optimal configuration with respect to the current SoL's sun path in order to maximize power generation. The \textit{\ac{DTE} Communication} mode follows after which the received traversing commands are processed during the \textit{Traverse} mode. A short time slot is alotted to science operations during the \textit{Science Stop - Short} mode so that multiple traverse Sols do not result in lack of scientific return, particularly for long multi-Sol traverse campaigns. The rover then assumes an optimal power generation pose during the \textit{Optimal Pose} mode to ensure maximum \ac{SA} power generation on the next Sol. Finally, the \textit{Idle - Night} mode is engaged during which the rover's survival is ensured with heaters that thermally regulate all critical systems.

The mechanical feasibility of achieving $\beta_{opt}$ or $\beta_{best}$ does not imply operational feasibility with respect to other subsystems. For instance, pitching the rover may hinder communication \ac{LOS} or produce thermal imbalances that would complicate heating the rover's critical systems. The implications are a systems engineering problem that is beyond the scope of this study. The assumption was made that all rover subsystems are designed to functional nominally for $\beta_{opt}$ and $\beta_{best}$.

%\subsection{Science Sol}
%\label{sec:ReferenceSols:ScienceSol}
%The Science Sol maximizes science data collection. The mode sequence is shown in Table \ref{tab:mission-scenario-science-sol}. It only differs from the Traverse Sol with a \textit{Science Stop - Long} mode instead of a \textit{Traverse} mode.
%\begin{table}[h]
\small
\centering
\caption{Science Sol mission scenario.}
\label{tab:mission-scenario-science-sol}
\begin{tabular}{l|l|c|c|}
\hline
\multicolumn{1}{|l|}{\multirow{2}{*}{\textbf{Mode}}} & \multirow{2}{*}{\textbf{Description}} & \multicolumn{2}{c|}{\textbf{Duration {[}min{]}}} \\ \cline{3-4}
\multicolumn{1}{|l|}{} &  & \multicolumn{1}{l|}{\textbf{Iani Chaos}} & \multicolumn{1}{l|}{\textbf{Ismenius Cavus}} \\ \hline
\multicolumn{1}{|l|}{\textbf{Idle - Day}} & \begin{tabular}[c]{@{}l@{}}Pre-Heating (2 hr).\\ UHF Communication (7 min).\\ Battery Charging.\end{tabular} & 88 & 88 \\ \hline
\multicolumn{1}{|l|}{\textbf{DTE Communication}} & \begin{tabular}[c]{@{}l@{}}Downlink (5 min).\\ Uplink (30 min).\end{tabular} & 35 & 35 \\ \hline
\multicolumn{1}{|l|}{\textbf{Science Stop - Long}} & Science activities. & 240 & 240 \\ \hline
\multicolumn{1}{|l|}{\textbf{Idle - Night}} & \begin{tabular}[c]{@{}l@{}}UHF Communication (7 min).\\ Thermal Regulation.\end{tabular} & 88 & 88 \\ \hline
 & \multicolumn{1}{r|}{\textbf{Total Duration {[}min{]}}} & \textbf{999} & \textbf{999} \\ \cline{2-4}
\end{tabular}
\end{table}


%\clearpage
%\subsection{Battery Recharge Sol}
%\label{sec:ReferenceSols:BatteryRechargeSol}
%The Battery Recharge Sol dedicates an entire Martian day to recharging the rover's batteries. The sequence and duration of its modes are shown in Table \ref{tab:mission-scenario-science-sol}. Only thermal and communication operations are executed during this Sol so that power draws from the battery are minimized in order to prioritize attaining the target charge state. No rover repositioning is required to adjust the \ac{SA} $\beta$ angle as it will have already been set to $\beta_{optimal}$ or $\beta_{best}$ at the end of a previous Sol.

%\begin{table}[h]
\footnotesize
\centering
\caption{Battery Charging Sol mission scenario.}
\label{tab:mission-scenario-battery-charging-sol}
\begin{tabular}{l|l|c|c|}
\hline
\multicolumn{1}{|l|}{\multirow{2}{*}{\begin{tabular}[c]{@{}l@{}}\\\textbf{Mode}\\\end{tabular}}} & \multirow{2}{*}{\begin{tabular}[c]{@{}l@{}}\\\textbf{Description}\\\end{tabular}} & \multicolumn{2}{c|}{\textbf{Duration {[}min{]}}} \\ \cline{3-4}
\multicolumn{1}{|l|}{} &  & \multicolumn{1}{l|}{\begin{tabular}[c]{@{}c@{}}\textbf{Iani}\\\textbf{Chaos}\end{tabular}} & \multicolumn{1}{l|}{\begin{tabular}[c]{@{}c@{}}\textbf{Ismenius}\\\textbf{Cavus}\end{tabular}} \\ \hline
\multicolumn{1}{|l|}{\textbf{Idle - Day}} & \begin{tabular}[c]{@{}l@{}}Pre-Heating (2 hr).\\ UHF Communication (7 min).\\ Battery Charging.\end{tabular} & 88 & 88 \\ \hline
\multicolumn{1}{|l|}{\textbf{Idle - Night}} & \begin{tabular}[c]{@{}l@{}}UHF Communication (7 min).\\ Thermal Regulation.\end{tabular} & 88 & 88 \\ \hline
 & \multicolumn{1}{r|}{\textbf{Total Duration {[}min{]}}} & \textbf{999} & \textbf{999} \\ \cline{2-4}
\end{tabular}
\end{table}


\subsection{Hibernation Sol}
\label{sec:ReferenceSols:HibernationSol}
The Hibernation Sol is the rover's survival mode during a dust storm. In this setting the rover solely draws power from its battery hence only the heater and a timer are on in order to conserve energy.

\begin{table}[h]
\footnotesize
\centering
\caption{Reference Sol for hibernation.}
\label{tab:mission-scenario-hibernation-sol}
\begin{tabular}{|l|l|}
\hline
\textbf{Mode} & \textbf{Description} \\ \hline
\textbf{Hibernation - Day} & \begin{tabular}[c]{@{}l@{}}All day.\\ Battery draw kept to a minimum.\end{tabular} \\ \hline
\textbf{Hibernation - Night} & \begin{tabular}[c]{@{}l@{}}All night.\\ Battery draw kept to a minimum.\end{tabular} \\ \hline
\end{tabular}
\end{table}


\section{Summary And Conclusion}
\label{sec:ReferenceSols:SummaryAndConclusion}
The chapter presents selected rover's modes which were used to define reference Sols. Of particular interest is the \textit{Tilt} mode which takes advantage of the rover's active suspect system to maximize power generation and battery charging during the \textit{Idle - Day} mode. Distinctions were made for the duration of the \textit{Traverse} mode with respect available daylight time at the different mission site locations as well as best and worst case terrain inclination.

% Operation durations were established for each mode as a prerequesite to determining power budgets in the following chapter.
