\vspace{-3ex}

The SherpaTT rover is prepared for further autonomous long distance traverses in terrain akin to the Martian environment. However, it features a fueled power generator which cannot be employed in extra-terrestial scenarios. As the rover is meant to approach a higher technology readiness level, a photovoltaic power subsystem is proposed to guide future design iterations. This thesis presents the solar array sizing, design, and integration processes considered for two Martian mission sites: Iani Chaos and Ismenius Cavus. An alternative use case for the active suspension system is presented so that the proposed solar arrays may be oriented and inclined into power generating configuration that are more favourable that what can be achieved with a horizontal surface.
