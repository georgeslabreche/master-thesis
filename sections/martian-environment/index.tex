%\input{sections/martian-environment/references.tex}



\section{Dust}
\label{sec:MartianEnvironment:Dust}
%\input{sections/s.tex}

% Martian Season - start with images
% Regolith, dust, and craters, other elements
% Settle with locations DTMS

Great dust storms (area > 10e6 km2) occur with a yearly probability of 30\% to 80\%. \citepower{Kerslake1999}

Local dust storms (area < 10e6 km2) occur with a 5\% probability in Mars equatorial regions and have only a minor impact on seasonal insulation due to their limited size, duration (a few days), and moderate OD (~1). \citepower{Kerslake1999}

the dust accumulation rate is assumed to be 5\% of that measured by Pathfinder \citepower{Kerslake1999}

Losses from lander vehicle shadowing and terrain masking are not yet modeled pending better definitions of lander configuration and landing sites. \citepower{Kerslake1999}

For desirable near-equatorial landing sites (not in canyons), shadowing and terrain masking losses will be small. This is due to high sun angles (that create short shadows) and the large component of diffuse solar insolation near dusk and dawn (when the terrain masking effect is largest). \citepower{Kerslake1999}


\section{Dust Storms}
\label{sec:MartianEnvironment:DustStorms}
%\input{sections/s.tex}
