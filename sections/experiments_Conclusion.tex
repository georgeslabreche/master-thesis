This chapter presents the \electromech system design of the two rover versions Sherpa and SherpaTT.
General design decisions valid for both systems are presented with the main influences resulting from the respective \ac{MRS}.
Apart from the kinematic design of the suspension systems, the central power management is discussed as this is a central part for the rovers to be a fully functional subsystem of a modular \ac{MRS}.

A manipulation arm for both rovers is developed using an evolutionary algorithm for morphology optimization.
Several use-cases are defined and a trajectory for the arm is built to test the use-cases in a physical simulation for fitness evaluation of the respective individuum.
The arm is then manufactured following a biologically inspired manufacturing methodology.
%Both rover systems make use of the same manipulation arm.

Comparing the main features of both implemented rover systems shows basically the same leg length of the first and second suspension generation, \refTab{tab:Design:ComparisonSherpaSherpaTT}.
The mass per leg as well as the total system mass are nearly identical as well.
With five \ac{DoF} per leg, the second generation suspension has one \ac{DoF} less per leg.
Horizontal and vertical \ac{LEP} movements in Sherpa are coupled, while they are independent in SherpaTT.
Note that the vertical and horizontal stroke listed for SherpaTT in the table are those resulting from the currently set software-joint limits as also shown in \refFig{fig:SherpaTT_Workspace-2d}.
Exploiting the full mechanical range as provided in \refTab{tab:AntriebsdatenSherpaTT} results in 860\unitmm vertical and 629\unitmm horizontal stroke.

\begin{table}[h]
%\vspace{-2ex}
  %\centering
  \hypersetup{hidelinks=true}
  %
  \caption[Comparison of main features of suspension system generations and rovers]{Comparison of main features of suspension system generations and rovers.
  %\dissExtraCaption{The stow footprint is measured using the \acp{LEP}, while the stow volume is a cuboid enclosing all parts of the rover, including those overhanging the \ac{LEP}'s footprint.}
  }
  \label{tab:Design:ComparisonSherpaSherpaTT}
  \begin{footnotesize}
      \begin{tabular}{l| rrr r ll r rr}
        \toprule
         \rowcolor{tableheadingcolor}
        System  & Leg length & Mass    & Mass & \ac{DoF}         & vert. & horz. & min stow  & \multicolumn{2}{c}{compactness}\\
        \rowcolor{tableheadingcolor}
                & zero pose &  (leg)  & total  & (leg)   & stroke &  stroke &volume    & footprint & volume \\
        \midrule
        \cellcolor{tablesubheadingcolor}Sherpa      & 976\unitmm & 25\unitkg & 160\unitkg &6          & 900\unitmm
%        \color{captionTextColor}{$^{\ast}$}
        & 260\unitmm
%        \color{captionTextColor}{$^{\ast\ast}$}
        &  2.24\unitcubmeter & 0.40 & 0.64\\
        \cellcolor{tablesubheadingcolor}SherpaTT\hspace{-1mm}    & 977\unitmm & 26\unitkg & \massSherpaTT &5          & 775\unitmm & 485\unitmm & 1.67\unitcubmeter & 0.79 & 0.72\\
        \bottomrule
      \end{tabular}
  \end{footnotesize}
  %\begin{scriptsize}
       % \vspace{-2mm}
%        \begin{flushleft}
%            \color{captionTextColor}{
%                $^{\ast}$ ~~~only possible with simultaneously changing horizontal position
%            \newline
%                $^{\ast\ast}$ ~\,only possible with simultaneously changing vertical position
%            %\newline
%%                $^{\ast\ast\ast}$ software safety limitation of IL and OL actuators, full range with mechanical limits: 860\unitmm and 629\unitmm see \refTab{tab:AntriebsdatenSherpaTT}.
%            }
%    \end{flushleft}
%    \end{scriptsize}
    %\vspace{-3ex}
\end{table}

Due to improved arrangement of the \ac{DoF}, the minimum stow volume is reduced from 2.24\unitcubmeter to 1.67\unitcubmeter.
This is also reflected in the values of compactness provided in the table; these values are based on the isoperimetric quotient:
The ratio of the area of the footprint to the area of a circle with the same perimeter is built.
A circle is the most compact shape in two-dimensional space,
therefore, the ratio ranges between 0 and 1 with high compactness being close to 1.
Similarly the compactness of the three-dimensional envelope volume is calculated using a sphere as most compact volume.
In both cases, the new design achieves higher compactness values.
%Higher compactness bears the potential for better fitting into a lander structure for transfer to Moon or Mars.



Apart from the compactness, the new arrangement of the \ac{DoF} allows to place the \ac{LEP} in a three-dimensional workspace compared to a two-dimensional spherical surface.
This design generates internal mobilities, that can be used to facilitate deployment and pickup of immobile elements in the \ac{MRS}.
Furthermore, the position and orientation of the central body in the support polygon can be changed without moving the wheels over the ground, which is beneficial for \ac{CoG} relocations in intricate slopes with low traction surface material.

A six \ac{DoF} \ac{FTS} is present in each leg of the new suspension system design.
Direct ground contact force measurement becomes available with this sensor, which in turn allows improved load balancing between the rover's ground contact points.

To conclude, with basically the same dimensioning (size, weight), superior properties are presented with the second suspension system design.
The subsequent chapters focus on the control and evaluation of this new suspension system and the rover SherpaTT.

