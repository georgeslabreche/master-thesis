\section{Introduction}
\label{sec:RequirementsAndDesignDrivers:Introduction}
Solar array design requirements were obtained from the following constraints:

\begin{itemize}
    \item Solar insolation.
    \item Propulsion power draws.
    \item Science instrument power budget.
    \item Thermal power budget.
\end{itemize}

The first two constraints were extracted from detailed analysis in Sections \ref{sec:PowerAndEnergyPredictions} and \ref{sec:PropulsionPowerConstraints}, respectively. The latter two are briefly presented in this section and extracted from literature and initial assumptions. Satisfying these requirements drove the desiye of the rover's solar array in terms or solar cell coverage area.

\section{Worst Cases}
\label{sec:RequirementsAndDesignDrivers:WorstCases}
Two solar insolation worst-cases needed to be defined: for a clear day and for a dust storm day.
Design drivers were based on worst-case scenarios for power and energy generation predictions as well as power draws for rover operations.

\subsection{Propulsion}
\label{sec:RequirementsAndDesignDrivers:WorstCases:Propulsion}

\subsection{Instrumentation}
\label{sec:RequirementsAndDesignDrivers:WorstCases:Instrumentation}

\subsection{Thermal}
\label{sec:RequirementsAndDesignDrivers:WorstCases:Thermal}

\section{Rover Modes}
\label{sec:RequirementsAndDesignDrivers:RoverModes}

\section{Assumptions}
\label{sec:RequirementsAndDesignDrivers:Assumptions}

\section{Requirements}
\label{sec:RequirementsAndDesignDrivers:Requirements}

\section{Conclusion}
\label{sec:RequirementsAndDesignDrivers:Conclusion}
