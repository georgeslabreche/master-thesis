Requirements and design drivers constrained the rover's \ac{SA} design. Initial assumptions on \ac{SA} performance degredation as well as propulsion power draw were made in Chapters \ref{sec:PowerAndEnergyPredictions} and \ref{sec:PowerBudget}, respectively. This chapter briefly introduces the selected solar cell technology and incorporates further assumptions pertaining to their \ac{BOL} and \ac{EOL} efficiencies.

The chapter is structured as follows: Section \ref{sec:RequirementsAndDesignDrivers:Assumptions} provides the assumptions. Requirements are presented in Section \ref{sec:RequirementsAndDesignDrivers:Requirements}. Section  \ref{sec:RequirementsAndDesignDrivers:DesignDrivers} presents the design drivers. The chapter is then summarized in Section \ref{sec:RequirementsAndDesignDrivers:SummaryAndConclusion}.


\section{Assumptions}
\label{sec:RequirementsAndDesignDrivers:Assumptions}
The following assumptions were identified:

\begin{itemize}
    \item[\textbf{A-01}] Propulsion power draw for flat terrain traverses is \SI{75}{\watt}.
    \item[\textbf{A-02}] Propulsion power draw for upslope terrain traverses up to \SI{30}{\degree} inclination is \SI{150}{\watt}.
    \item[\textbf{A-03}] Solar cell technology is \ac{IMM} solar cell with AM0 efficiency of 32.3\% at \ac{BOL} and 22.5\% at \ac{EOL} \citepower{Sharps2017}.
    \item[\textbf{A-04}] IMM Solar cell efficiency on Mars surface is 22\% at \ac{EOL}.
    \item[\textbf{A-05}] Solar cell packing efficiency is 85\%.
    \item[\textbf{A-06}] Solar array efficiency varies by 3\% due to changing temperature and red-shift spectral losses through the day time-period \citepower{Kerslake1999}.
    \item[\textbf{A-07}] Solar array dust performance degradation saturates at 30\% \citepower{Stella2005}.
    \item[\textbf{A-08}] Solar array performance degradation from shadowing by protruding rover structures is 5\%.
\end{itemize}

\clearpage
\section{Requirements}
\label{sec:RequirementsAndDesignDrivers:Requirements}
The requirements were narrowed down based on the following considerations:
\begin{itemize}
    \item Interest in long traverses from the PERASPERA programme on space robotics \citeother{PERASPERA}.
    \item Focus on space robotics development for long autonomous traverses at \ac{DFKI} \citeother{OG6} \citeother{OG10}.
    \item Mission site at Iani Chaos is approximately \SI{100}{\kilo\meter} from closest resource deposits of interest.
    \item Area of interest at Ismenius Cavus is approximately $\SI{100}{\kilo\meter} \times \SI{50}{\kilo\meter}$.
    \item Dust storm tracking in \citemarsenv{Battalio2019} observed that the average duration of local dust storms is approximately 7 Sols.
\end{itemize}

Thus, for the purpose of initial \ac{SA} design, the following \ac{EOL} requirements were identified:

\begin{itemize}
    \item[\textbf{R-01}] The rover shall be able to traverse at least \SI{100}{\kilo\meter} of flat surface during a Martian year.
    \item[\textbf{R-02}] The rover shall be able to traverse inclined surfaces of up to \SI{30}{\degree} at optical depths of up to $\tau = 1$.
    \item[\textbf{R-03}] The rover shall be able to traverse during any Sol in a Martian year at optical depths of up to $\tau = 1$.
    \item[\textbf{R-04}] The rover shall be able to survive dust storms at optical depths $\tau = 2$ for at least 7 Sols.
\end{itemize}


\section{Design Drivers}
\label{sec:RequirementsAndDesignDrivers:DesignDrivers}
The following \ac{SA} design drivers were identified:

\begin{itemize}
    \item[\textbf{DD-01}] Limited amount of unfolding during deployment.
    \item[\textbf{DD-02}] Manipulator arm's end effector can access the ground.
    \item[\textbf{DD-03}] Movement range of the suspension system is unobstructed.
    \item[\textbf{DD-04}] Position of \ac{CoG} is within the rover's body.
\end{itemize}

\section{Summary and Conclusion}
\label{sec:RequirementsAndDesignDrivers:SummaryAndConclusion}
The assumptions, requirements, and design drivers presented in this chapter were applied for both mission sites.
