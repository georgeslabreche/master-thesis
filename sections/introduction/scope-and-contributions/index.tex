This thesis builds upon the areas of interest elaborated in previous work. Notably, it acknowledges that the rover is intended to be part of a heterogeneous modular \ac{MRS} and that a wealth of data has been collected over the course of multiple field test campaigns. Radical system changes are avoided and design constraints are driven by the collected performance measurements. Design changes made to accomodate the proposed \ac{PV} system will be mindful of mission objectives inherent with scenarios such as crater exploration and operating within a sample return logistics chain.

The main contribution of this thesis is the design of \acp{SA} for the rover at two Mars mission sites. These sites are selected so that the distinct requirements between different solar radiation environments may be appreciated. Re-usability of the implemented solutions is taken into account throughout the entire development of the thesis. As such, formulas and calculations pertaining to solar radiation as well as power and energy predictions have been packged into a thoroughly documented general purpose R library. Furthermore, a rudimentary power calculation plugin is developed as a prototype for the MARS real-time robotic simulation and visualisation tool. This plugin offers insights on how the Blender/Phobos robot modeling software suite may be extended to support modeling solar panels as robot sensors.

Further contributions pertain to formalizing requirements and constraints that need to be considered for future solar powered iterations of the rover towards reaching higher \acp{TRL}. Beyond the Martian environment, the proposed solution serves as a template to integrating solar powered systems for other celestial bodies such as in terrestial and lunar mission scenarios.

Finally, this thesis provides an alternative use case for the rover's active suspension system. Thus far, explored opportunities granted by such a system have been restricted to analyzing its benefits with respect to traversing challenging topographies. Introducing how such a system can be leveraged for other aspects of mission planning broadens the field of research.
