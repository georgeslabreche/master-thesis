The rover SherpaTT has been deployed in several field experiments where it was put under test within natural and unstructured Mars analogue terrain with respect to general morphology and geology. The rover displayed the ability to cope with natural terrain and to fulfill the task of being an exploration and sampling rover. Currently, the rover is prepared for further autonomous long distance traverses in terrain akin to the Martian environment. However, it features a fueled power generator which cannot be employed in Martian missions thus limiting the system's autonomous mission lifetime to the discharge rate of its two LiPo batteries.

As the rover is meant to approach a higher \ac{TRL}, further development is required on its electrical power subsystem if it is to operate in long term missions. The thesis explores \ac{SA} configurations for the a Mars environment in order to guide future design iterations to navigate the topography of this planetary surface. The constraints imposed from the active suspension system with flexible footprints and varying heights of structural parts of the legs are considered in the design phase.

Initial \ac{SA} sizing requirements are derived from Mars mission sites, Iani Chaos and Ismenius Cavus, that will impose power storage and consumption constraints based on available daily insolations. The \ac{SA} design will be driven by the use the rover's active suspension system as a solar tracking mechanism. An alternative use of the wheeled leg system is thus presented for a use case that goes beyond the obvious scenario of negotiating complex terrains such as steep slopes. Specifically, the findings demonstrate traverse and mass reductions gains that are obtained with a suspension system driven orientation and inclination capable \ac{SA} surface when compared to sizing constraints inherent to a horizontal configuration.


%Sites with high geological and exobiological science potential typically require navigating complicated terrains, such as climbing up mountains and volcanoes or going down valleys and craters, which offer an opportunity to analyze solar energy production on inclined surfaces.

%as a function of geographic latitude, areocentric longitude, atmospheric opacity, dust deposition, and sun angle of incident on the \ac{SA}



%A Lunar mission scenario will also be investigated with particular interest on access to permanently shadowed areas in the southern pole. The mission scenarios will put particular emphasis on including topograpy best tackled by the rover's wheeled leg design, such as climbing up mountains and volcanoes or going down valleys and craters. Sites with high geological and exobiological science potential typically require navigating complicated terrains which offer an opportunity for further power budget analysis that take into account solar irradiance on slopes.
