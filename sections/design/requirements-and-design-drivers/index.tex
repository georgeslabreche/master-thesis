Requirements and design drivers constrained the rover's \ac{SA} design. Initial assumptions on \ac{SA} performance degredation as well as propulsion power draw were made in Chapters \ref{sec:PowerAndEnergyPredictions} and \ref{sec:PowerBudget}, respectively. This chapter briefly introduces the selected solar cell technology and presents assumptions pertaining to their \ac{BOL} and \ac{EOL} efficiencies. Further assumptions are made regarding the rover's mobility capabilities.


\subsection{Assumptions}
\label{sec:RequirementsAndDesignDrivers:Assumptions}
The following assumptions were identified:

\begin{enumerate}[label=\textbf{\textcolor{BulletBlue}{A-\arabic*}}]
    \item Propulsion power draw for flat terrain traverses is \SI{75}{\watt}.
    \item Propulsion power draw for upslope terrain traverses up to \SI{30}{\degree} inclination is \SI{150}{\watt}.
    \item Solar cell technology is \ac{IMM} solar cell with AM0 efficiency of 32.3\% at \ac{BOL} and 22.5\% at \ac{EOL} \citepower{Sharps2017}.
    \item \label{itm:ass:solar_cell_efficiency} \ac{IMM} Solar cell efficiency on Mars surface is 22\% at \ac{EOL}.
    \item \label{itm:ass:packing_efficiency} Solar cell packing efficiency is 85\%.
    \item \label{itm:ass:red_shifts} Solar array efficiency varies by 3\% due to changing temperature and red-shift spectral losses through the day time-period \citepower{Kerslake1999}.
    \item \label{itm:ass:dust_deposition_saturation} Solar array dust performance degradation saturates at 30\% \citepower{Stella2005}.
    \item \label{itm:ass:protruding_shadowing} Solar array performance degradation from shadowing by protruding rover structures is 5\%.
    \item \label{itm:ass:sa_surface_density} The \ac{SA} surface density is \SI{3.7}{kg.m^{-2}}.
    \item The rover's nominal velocity is at least \SI{2}{cm.s^{-2}}.
    \item There is an average 15\% slippage when traversing a flat terrains for any given distance.
    \item There is an average 30\% slippage when traversing an upslope terrain for any given distance and slope angle.
\end{enumerate}

\todo[inline]{\textbf{TODO:} Reference past missiosn for A-09 and Cordes 2019 for A-11 and A-12. Is slippage still as high as 15\%-30\% if the system is programmed to stop driving when slippage is detected?}

\subsection{Requirements}
\label{sec:sec:Design:RequirementsAndDesignDrivers:Requirements}
The requirements were narrowed down with the following considerations:

\begin{enumerate}[label=\textbf{\textcolor{BulletBlue}{(\alph*)}}]
    \item Interest in long traverses from the PERASPERA programme on space robotics \citeother{PERASPERA}.
    \item Focus on space robotics development for long autonomous traverses at \ac{DFKI} \citeother{OG6} \citeother{OG10}.
    \item Mission site at Iani Chaos is approximately \SI{100}{\kilo\meter} from closest resource deposits of interest.
    \item Area of interest at Ismenius Cavus is approximately $\SI{100}{\kilo\meter} \times \SI{50}{\kilo\meter}$.
    \item Dust storm tracking in \citemarsenv{Battalio2019} observed that the average duration of local dust storms is approximately seven Sols.
\end{enumerate}

Thus, for the purpose of initial \ac{SA} design, the following \ac{EOL} requirements were identified for a mission life time of one \ac{MY}:

\begin{enumerate}[label=\textbf{\textcolor{BulletBlue}{R-\arabic*}}]
    %\item The rover shall be able to traverse flat terrain at an optical depth of $\tau = 1$.
    %\item The rover shall be able to traverse \SI{30}{\degree} inclined terrain at an optical depth of $\tau = 1$.
    \item \label{itm:req:total_distance_flat_terrain} The rover shall traverse flat terrain for a total distance of at least \SI{20}{\kilo\meter}.
    \item \label{itm:req:survive_tau1} The rover shall survive optical depths of up to $\tau = 1$.
    \item \label{itm:req:survice_tau2} The rover shall survive optical depths $\tau = 2$ for at least seven Sols.
\end{enumerate}

\subsection{Constraint}
\label{sec:Design:RequirementsAndDesignDrivers:Constraints}
The design had to allow \textit{non-Traverse Sols}, such as for long science stops, intermittent hibernation periods during dust storms, battery recharging periods, or events of limited activities such as during a Solar Conjunction Break. Furthermore, traverses may only occur during daylight to allow terrain observations to be made in case of slip and skids that pass a yet to be defined threshold. For the scope of this study, the following constraints were considered rather than formalizing requirements for the implicated subsystems:

\begin{enumerate}[label=\textbf{\textcolor{BulletBlue}{C-\arabic*}}]
    \item There shall be a minimum of at least two \textit{non-Traverse Sols} between each \textit{Traverse Sol}.
    \item \label{itm:con:daylight_traverse} The rover can be in \textit{Traverse mode} only during daylight.
\end{enumerate}

\subsection{Design Drivers}
\label{sec:Design:RequirementsAndDesignDrivers:DesignDrivers}
The following \ac{SA} design drivers were identified:

\begin{enumerate}[label=\textbf{\textcolor{BulletBlue}{D-\arabic*}}]
    \item Limited amount of unfolding during deployment.
    \item Manipulator arm's end effector can access the ground.
    \item Manipulator arm can access four \ac{PLI} stored in the rover.
    \item Movement range of the suspension system is unobstructed.
    \item Position of \ac{CoG} is within the rover's body.
\end{enumerate}

\subsection{Summary}
\label{sec:Design:RequirementsAndDesignDrivers:Summary}
\todo[inline]{\textbf{TODO:} Write section summary.}
%The assumptions, requirements, and design drivers presented in this chapter were applied for both mission sites.
