\todo[inline]{\textbf{TODO:} Write section introduction.}

\subsection{Sizing}
\todo[inline]{\textbf{TODO:} Short introduction.}

\subsubsection{Solar Array}
Equation \ref{eq:SA_slope_energy} is rearranged in into an expression of the solar cell coverage area $A$, shown in Equation \ref{eq:solar_cell_coverage_area}:

\begin{equation}
  \label{eq:solar_cell_coverage_area}
  A = \frac{E}{\eta \cdot H_{\beta} \cdot PR}
\end{equation}

where $E$ becomes the energy required by the rover on that Sol and $H_{h}$ the worst-case available daily insolation, henceforth $E_{req}^{worst}$ and $H_{\beta}^{worst}$ respectively. $H_{\beta}^{worst}$ is used instead of $H_{h}^{worst}$ in order to take advantage of the rover's active suspension system to generate the most available worst case daily insolation. The values for these varibles are taken from assumptions and previously calculated energies and daily insolations:

\begin{enumerate}[label=\textbf{\textcolor{BulletBlue}{(\alph*)}}]
    \item From Table \ref{tab:worst-case-traverse-sol-power-budget}, $E_{req}^{worst}$ is \SI{775}{\watt\hour} at Iani Chaos and \SI{750}{\watt\hour} at Ismenius Cavus.
    \item From Tables \ref{tab:insolation-iani-chaos-clear-and-dusty-days} and \ref{tab:insolation-ismenius-cavus-clear-and-dusty-days}, $H_{\beta}^{worst}$ for $\tau=1$ is \SI{2479}{Wh.m^{-2}} at Iani Chaos and \SI{1345}{Wh.m^{-2}} at Ismenius Cavus.
    \item From \ref{itm:ass:red_shifts}, \ref{itm:ass:dust_deposition_saturation}, and \ref{itm:ass:protruding_shadowing}, \ac{PR} at \ac{EOL} is $PR_{EOL} = 1 - (0.03 + 0.3 + 0.05) = 0.62$.
    \item From \ref{itm:ass:solar_cell_efficiency}, $\eta_{EOL} = 0.22$.
\end{enumerate}


The required solar cell coverage area at Iani Chaos then becomes:
\begin{align}
  \label{calc:solar_cell_area_ismenius_cavus}
  A_{iani} &= \frac{E_{req}^{worst}}{\eta_{EOL} \cdot H_{\beta}^{worst} \cdot PR_{EOL}}\\
           &= \frac{775}{0.22 \cdot 2479 \cdot 0.62}\\
           &= \SI{2.29}{m^{2}}
\end{align}

and at Ismenius Cavus:
\begin{align}
  \label{calc:solar_cell_area_iani_chaos}
  A_{ismenius} &= \frac{E_{req}^{worst}}{\eta_{EOL} \cdot H_{\beta}^{worst} \cdot PR_{EOL}}\\
               &= \frac{750}{0.22 \cdot 1345 \cdot 0.62}\\
               &= \SI{4.09}{m^{2}}
\end{align}

At Iani Chaos, from \ref{itm:ass:packing_efficiency} the resulting \ac{SA} area is  \SI{2.7}{m^{2}} and from \ref{itm:ass:sa_surface_density} its mass is 9.95 \si{\kilo\gram}. Taking advantage of \ac{SA} inclination capabilities with $\beta_{best} = \SI{10}{\degree}$ resulted in \ac{SA} sizing decrease of \SI{3.9}{\percent} when compared with a horizontal surface configuration. For $\tau = 1$ during global dust storm season and $\tau = 0.4$ during the remainder of the year, the total maximum flat traverse distance achievable over the course of one \ac{MY} is increased by \SI{8.79}{\percent} from \SI{59.13}{\kilo\meter} to \SI{64.33}{\kilo\meter}.

At Ismenius Cavus, the resulting \ac{SA} area and mass are \SI{4.8}{m^{2}} at 17.8 \si{\kilo\gram}. The \ac{SA} sizing decrease is \SI{4.6}{\percent}. The one \ac{MY} maximum achievable flat traverse distance is increased by \SI{1.39}{\percent} from \SI{67.4}{\kilo\meter} to \SI{68.34}{\kilo\meter}.

The traverse distance gains attributed to \ac{SA} inclination capabilities do not seem to justify adopting the complexities of an active suspension system for the purpose of increasing traverse distance via solar tracking. This is particularly true at Ismenius Cavus. The savings in \ac{SA} surface area and mass also leave much to be desired. To explain the lack of significant traverse distance gain with $\beta_{best}$, the generated \ac{SA} energy and  maximum traverse durations are plotted in Figures \ref{fig:plot:iani-chaos-generated-energy-and-max-traverse-durations} and \ref{fig:plot:ismenius-cavus-generated-energy-and-max-traverse-durations}.

\clearpage
\begin{figure}[h]
\captionsetup[subfigure]{justification=centering}
\vspace{-2ex}
	\centering
    %% setup sizes
    \setlength{\subfigureWidth}{0.50\textwidth}
    \setlength{\graphicsHeight}{80mm}
    %% kill hyper-link highlighting
    \hypersetup{hidelinks=true}%
    %% the figures
    \begin{subfigure}[t]{\subfigureWidth}
        \centering
        \includegraphics[height=\graphicsHeight]{sections/design/solar-array/plots/ianichaos-daily-generated-energy.png}
        \subcaption{Generated Energy}
        \label{fig:plot:sub:iani-chaos-generated-energy}
    \end{subfigure}\hfill
    \begin{subfigure}[t]{\subfigureWidth}
        \centering
        \includegraphics[height=\graphicsHeight]{sections/design/solar-array/plots/ianichaos-75w-max-traverse-durations.png}
  		\subcaption{Maximum Traverse Durations}
		\label{fig:plot:sub:iani-chaos-max-traverse-durations}
	\end{subfigure}\\[0.8ex]
    \caption[Generated energy and maxium achievable flat terrain traverse durations at Iani Chaos]
            {Generated energy and maxium achievable flat terrain traverse duration at Iani Chaos. Optical depth  $\tau = 1$ was used for global dust storm season ($\SI{185}{\degree} \leq L_{s} \leq \SI{315}{\degree}$) and $\tau = 0.4$ for the remainder of the year. The \textit{available daylight traverse time} corresponds to the amount of daylight hours left in a \textit{Traverse Sol} after subtracting the time taken by non-Traverse modes: \textit{Idle - Day}, \textit{\ac{DTE} Communication}, \textit{Science Stop - Short}, and \textit{Optimal Pose}. The maximum traverse durations for \ac{SA} horizontal do not consider the \textit{Optimal Pose} mode.}
    \label{fig:plot:iani-chaos-generated-energy-and-max-traverse-durations}
\vspace{-2ex}
\end{figure}

\begin{figure}[h]
\captionsetup[subfigure]{justification=centering}
\vspace{-2ex}
	\centering
    %% setup sizes
    \setlength{\subfigureWidth}{0.50\textwidth}
    \setlength{\graphicsHeight}{80mm}
    %% kill hyper-link highlighting
    \hypersetup{hidelinks=true}%
    %% the figures
    \begin{subfigure}[t]{\subfigureWidth}
        \centering
        \includegraphics[height=\graphicsHeight]{sections/design/solar-array/plots/ianichaos-75w-traverse-gains-for-27m2-sa-area.png}
        \subcaption{\ac{SA} area = \SI{2.7}{\meter\squared}}
        \label{fig:plot:sub:iani-chaos-flat-traverse-gains-for-initial-sa-area}
    \end{subfigure}\hfill
    \begin{subfigure}[t]{\subfigureWidth}
        \centering
        \includegraphics[height=\graphicsHeight]{sections/design/solar-array/plots/ianichaos-75w-traverse-gains-for-different-sa-areas.png}
  		\subcaption{For different \ac{SA} areas}
		\label{fig:plot:sub:iani-chaos-flat-traverse-gains-for-different-sa-area}
	\end{subfigure}\\[0.8ex]
    \caption[Flat traverse distance gains at Iani Chaos]
            {Flat traverse distance gains at Iani Chaos.}
    \label{fig:plot:iani-chaos-flat-traverse-gains}
\vspace{-2ex}
\end{figure}


\begin{figure}[h]
\captionsetup[subfigure]{justification=centering}
\vspace{-2ex}
	\centering
    %% setup sizes
    \setlength{\subfigureWidth}{0.50\textwidth}
    \setlength{\graphicsHeight}{80mm}
    %% kill hyper-link highlighting
    \hypersetup{hidelinks=true}%
    %% the figures
    \begin{subfigure}[t]{\subfigureWidth}
        \centering
        \includegraphics[height=\graphicsHeight]{sections/design/solar-array/plots/ismeniuscavus-daily-generated-energy.png}
        \subcaption{Generated Energy}
        \label{fig:plot:sub:ismenius-cavus-generated-energy}
    \end{subfigure}\hfill
    \begin{subfigure}[t]{\subfigureWidth}
        \centering
        \includegraphics[height=\graphicsHeight]{sections/design/solar-array/plots/ismeniuscavus-75w-max-traverse-durations.png}
  		\subcaption{Maximum Traverse Durations}
		\label{fig:plot:sub:ismenius-cavus-max-traverse-durations}
	\end{subfigure}\\[0.8ex]
    \caption[Generated energy and maxium achievable flat terrain traverse durations at Ismenius Cavus]
            {Generated energy and maxium achievable flat terrain traverse duration at Ismenius Cavus. Optical depth  $\tau = 1$ was used for global dust storm season ($\SI{185}{\degree} \leq L_{s} \leq \SI{315}{\degree}$) and $\tau = 0.4$ for the remainder of the year. The \textit{available daylight traverse time} corresponds to the amount of daylight hours left in a \textit{Traverse Sol} after subtracting the time taken by non-Traverse modes: \textit{Idle - Day}, \textit{\ac{DTE} Communication}, \textit{Science Stop - Short}, and \textit{Optimal Pose}. The maximum traverse durations for \ac{SA} horizontal do not consider the \textit{Optimal Pose} mode.}
    \label{fig:plot:ismenius-cavus-generated-energy-and-max-traverse-durations}
\vspace{-2ex}
\end{figure}

\begin{figure}[h]
\captionsetup[subfigure]{justification=centering}
\vspace{-2ex}
	\centering
    %% setup sizes
    \setlength{\subfigureWidth}{0.50\textwidth}
    \setlength{\graphicsHeight}{80mm}
    %% kill hyper-link highlighting
    \hypersetup{hidelinks=true}%
    %% the figures
    \begin{subfigure}[t]{\subfigureWidth}
        \centering
        \includegraphics[height=\graphicsHeight]{sections/design/solar-array/plots/ismeniuscavus-75w-traverse-gains-for-48m2-sa-area.png}
        \subcaption{\ac{SA} area = \SI{4.8}{\meter\squared}}
        \label{fig:plot:sub:ismenius-chaos-flat-traverse-gains-for-initial-sa-area}
    \end{subfigure}\hfill
    \begin{subfigure}[t]{\subfigureWidth}
        \centering
        \includegraphics[height=\graphicsHeight]{sections/design/solar-array/plots/ismeniuscavus-75w-traverse-gains-for-different-sa-areas.png}
  		\subcaption{For different \ac{SA} areas}
		\label{fig:plot:sub:ismenius-chaos-flat-traverse-gains-for-different-sa-area}
	\end{subfigure}\\[0.8ex]
    \caption[Flat traverse distance gains at Ismnenius Chaos]
            {Flat traverse distance gains at Ismnenius Chaos.}
    \label{fig:plot:ismenius-chaos-flat-traverse-gains}
\vspace{-2ex}
\end{figure}


\todo[inline]{\textbf{TODO:} Size SA with hibernation power budget. Reduce Hibernation mode power draw.}

%\label{itm:req:total_distance_flat_terrain}

\clearpage
\subsubsection{Battery}
\todo[inline]{\textbf{TODO:} Battery size based on energy required to keep the rover Warm through the night. Check if the calculated size satisfies the hibernation requirement. If not, resize.}

\subsection{Baseline Design}
\todo[inline]{\textbf{TODO:} Refer to design drivers and present solution.}

\subsection{Mechanisms}
\todo[inline]{\textbf{TODO:} Short introduction.}

\subsubsection{Deployment}
\todo[inline]{\textbf{TODO:} 1. Sequence and 2. Cantilever beam analysis at every deployment step.}

\subsubsection{HDRMs}
\todo[inline]{\textbf{TODO:} Calculate for Force acting on the \ac{HDRM}.}

\subsection{Summary}
\todo[inline]{\textbf{TODO:} Write section summary.}
