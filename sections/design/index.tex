This chapter elaborates on the power budget and requirements leading to the proposed solar array design. Reference Sols and their rover modes are defined in \refSec{sec:Design:ReferenceSols} so that mission scenario power budgets may be formalized in \refSec{sec:Design:PowerBudget}. Particular attention is put on analyzing field trial data from which propulsion power draw requirements are extracted. \refSec{sec:Design:RequirementsAndDesignDrivers} consolidates assumptions, requirements, conwtraints, and design drivers that will guide the proposed design of the rover's \ac{PV} system. Solar array sizing and configurations as well as necessary rover body redesign are presented in \refSec{sec:Design:SolarArray}. A rudimentary simulation environment for solar power management is introduced in \refSec{sec:Design:Simulation} as a potential for future work before summarizing and concluding the chapter in \refSec{sec:Design:SummaryAndConclusion}.

\section{Reference Sols}
\label{sec:Design:ReferenceSols}
Baseline content for this section were taken from \citeother{CDF2014} and constrained to the scope of this thesis. Adjustements were made in relation to the mission sites described in the previous chapter as well as with the rover's capabilities with respect to its active suspension system. Rover modes were identified and then sequenced into high-level reference Sols for mission planning. Defining rover modes and reference Sols served as a prerequisite to formalizing power budgets by assignment each rover modes with their power draw and minimal operational duration.

%The chapter is structured as follows: Section \ref{sec:ReferenceSols:RoverModes} provides general definitions of rover modes. The identified rover modes are sequenced into reference Sols in Section \ref{sec:ReferenceSols:ReferenceSols}. The chapter is then summarized in Section \ref{sec:PowerBudget:SummaryAndConclusion}.

\subsection{Rover Modes}
\label{sec:ReferenceSols:RoverModes}
Not all modes defined in \citeother{CDF2014} were included in this section, however; they may become of interest in future design iterations. The omitted modes are \textit{Launch}, \textit{\ac{EDL}}, \textit{Deployment}, \textit{Science Stop - Long},  and \textit{Safe} modes. The rover modes considered in this thesis are as follow:

\begin{itemize}
    \item \textbf{Traverse - Flat:} Pre-planned flat surface traverses to target destinations.
    \item \textbf{Traverse - Upslope:} Pre-planned upslope surface traverses to target destinations.
    \item \textbf{Science Stop - Short:} Short science activities in between traverses.
    \item \textbf{\ac{DTE} Communication:} Pre-planned communication sessions.
    \item \textbf{Idle - Day:} All day dedicated to charging batteries.
    \item \textbf{Idle - Night:} Minimal battery usage during night.
    \item \textbf{Hibernation:} Survival mode during dust storms ($\tau \geq 2$).
    \item \textbf{Optimal Pose:} Repositioning the rover along its principal yaw, pitch, and roll axes to maximized next Sol solar array power generation.
\end{itemize}
%\ac{DTE} Communication operations are encapsulated as their own mode whereas \ac{UHF} Communications are mode activities conducted during \textit{Traverse}, \textit{Science Stop}, and \textit{Idle} modes.

%Communication operations and their schedules were also taken from \citeother{CDF2014}:
%\begin{itemize}
%    \item Uplink of sol plan either early in the morning (\SI{30}{\minute} \ac{DTE} or \SI{7}{\minute} \ac{UHF}) or during the night (\SI{7}{\minute} \ac{UHF}).
%    \item Downlink of high priority data required for planning of the next sol either in the afternoon (\SI{5}{\minute} \ac{DTE}) or during the night (\SI{7}{\minute} \ac{UHF}).
%\end{itemize}

The \textit{Optimal Pose} mode was not taken from \citeother{CDF2014}. It was created for the purpose of this study so that the rover may use its active suspension system as a solar tracking mechanism to maximize \ac{SA} power generation. For the sake of simplicity, a constraint was applied to the pitch and roll axes which mutually excluded them from both being actuated. Thus, during the \textit{Optimal Pose} mode, only yaw-pitch or yaw-roll rotation combinations are allowed. The inclination angle resulting from a pith or roll rotations is the $\beta$ angle. The orientation angle with respect to the direction of the equator which resulting from a yaw rotation is the surface azimuth angle $\gamma_{c}$. These angles were introduced in Section \ref{sec:MartianEnvironment:SolarRadiation:InclinedSurface}.

Rover subsystems as well as their states during the rover modes were taken from \citeother{CDF2014} and are shown in Table \ref{tab:rover-modes}.

 %Some mode durations were adjusted from those presented in \citeother{CDF2014} in response to power budgeting constraints pertaining to the scope of this research.

\begin{table}[h]
\footnotesize
\centering
\caption[Rover modes]
    {Rover modes. DT/NT - Daytime/Night-time; BC - Battery Charging; P - Propulsion; N - Navigation; C - Communication; S - Science; H - Heaters. Duration is per Sol.}
\label{tab:rover-modes}
\begin{tabular}{|l|c|c|c|c|c|c|c|l|}
\hline
\textbf{Name} & \textbf{DT/NT} & \textbf{BC} & \textbf{P} & \textbf{N} & \textbf{C} & \textbf{S} & \textbf{H} & \textbf{Duration} \\ \hline
Traverse - Flat & DT & OFF & ON & ON & ON & OFF & OFF & Variable \\ \hline
Traverse - Upslope & DT & OFF & ON & ON & ON & OFF & OFF & Variable \\ \hline
Science Stop - Short & DT & ON & OFF & OFF & ON & ON & OFF & 60 min \\ \hline
\ac{DTE} Commnunication & DT & OFF & OFF & OFF & ON & OFF & OFF & 35 min \\ \hline
Idle - Day & DT & ON & OFF & OFF & ON & OFF & OFF & All day \\ \hline
Idle - Night & NT & OFF & OFF & OFF & ON & OFF & ON & All night \\ \hline
Hibernation & DT/NT & OFF & OFF & OFF & OFF & OFF & ON & All day/night \\ \hline
Optimal Pose & DT & ON & \multicolumn{1}{l|}{ON} & \multicolumn{1}{l|}{OFF} & \multicolumn{1}{l|}{ON} & \multicolumn{1}{l|}{OFF} & \multicolumn{1}{l|}{OFF} & 10 min \\ \hline
\end{tabular}
\end{table}




\subsection{Reference Sols}
\label{sec:ReferenceSols:ReferenceSols}
Flat and upslope traverses are the worst-case modes in terms of energy consumption at two different topographies. Hibernation must be constrained with respect to battery depletion in order to appreciate the rover's survivabilty during a dust storm. As such, only the \textit{Flat Traverse}, \textit{Upslope Traverse}, and \textit{Hibernation} reference Sols required analysis.


\subsubsection{Traverse Sol}
\label{sec:ReferenceSols:TraverseSol}
A \textit{Traverse Sol} allows the rover to reach a pre-planned target destination. Flat and upslope traverses have identical mode sequences, shown in Table \ref{tab:mission-scenario-traverse-sol}, only differing in the duration of their respective \textit{Traverse} modes. This distinction relates to propulsion power draw differences that are presented in Chapter \ref{sec:PowerBudget}.

\begin{table}[h]
\footnotesize
\centering
\caption{Reference Sol for flat or upslope traverse.}
\label{tab:mission-scenario-traverse-sol}
\begin{tabular}{|l|l|}
\hline
\textbf{Mode} & \textbf{Description} \\ \hline
\textbf{Idle - Day} & \begin{tabular}[c]{@{}l@{}}Pre-Heating.\\ UHF Communication.\\ Battery Charging.\end{tabular} \\ \hline
\textbf{DTE Communication} & \begin{tabular}[c]{@{}l@{}}Downlink.\\ Uplink.\end{tabular} \\ \hline
\textbf{Traverse - Flat or Upslope} & Propulsion to target destination. \\ \hline
\textbf{Science Stop - Short} & Science operations. \\ \hline
\textbf{Optimal Pose} & Maximize SA power generation for next Sol. \\ \hline
\textbf{Idle - Night} & \begin{tabular}[c]{@{}l@{}}UHF Communication.\\ Thermal Regulation.\end{tabular} \\ \hline
\end{tabular}
\end{table}


A Traverse Sol begins with the \textit{Idle - Day} mode, which is mostly dedicated to recharging the rover's batteries. On the previous Sol, the \ac{SA} was tilted and oriented into an optimal configuration with respect to the current SoL's sun path in order to maximize power generation. The \textit{\ac{DTE} Communication} mode follows after which the received traversing commands are processed during the \textit{Traverse} mode. A short time slot is alotted to science operations during the \textit{Science Stop - Short} mode so that multiple traverse Sols do not result in lack of scientific return, particularly for long multi-Sol traverse campaigns. The rover then assumes an optimal power generation pose during the \textit{Optimal Pose} mode to ensure maximum \ac{SA} power generation on the next Sol. Finally, the \textit{Idle - Night} mode is engaged during which the rover's survival is ensured with heaters that thermally regulate all critical systems.

The mechanical feasibility of achieving $\beta_{opt}$ or $\beta_{best}$ does not imply operational feasibility with respect to other subsystems. For instance, pitching the rover may hinder communication \ac{LOS} or produce thermal imbalances that would complicate heating the rover's critical systems. The implications are a systems engineering problem that is beyond the scope of this study. The assumption was made that all rover subsystems are designed to functional nominally for $\beta_{opt}$ and $\beta_{best}$.

%\subsection{Science Sol}
%\label{sec:ReferenceSols:ScienceSol}
%The Science Sol maximizes science data collection. The mode sequence is shown in Table \ref{tab:mission-scenario-science-sol}. It only differs from the Traverse Sol with a \textit{Science Stop - Long} mode instead of a \textit{Traverse} mode.
%\input{sections/design/reference-sols/tables/mission-scenario-science-sol.tex}

%\clearpage
%\subsection{Battery Recharge Sol}
%\label{sec:ReferenceSols:BatteryRechargeSol}
%The Battery Recharge Sol dedicates an entire Martian day to recharging the rover's batteries. The sequence and duration of its modes are shown in Table \ref{tab:mission-scenario-science-sol}. Only thermal and communication operations are executed during this Sol so that power draws from the battery are minimized in order to prioritize attaining the target charge state. No rover repositioning is required to adjust the \ac{SA} $\beta$ angle as it will have already been set to $\beta_{optimal}$ or $\beta_{best}$ at the end of a previous Sol.

%\begin{table}[h]
\footnotesize
\centering
\caption[Battery Charging Sol mission scenario]
    {Battery Charging Sol mission scenario.}
\label{tab:mission-scenario-battery-charging-sol}
\begin{tabular}{l|l|c|c|}
\hline
\multicolumn{1}{|l|}{\multirow{2}{*}{\begin{tabular}[c]{@{}l@{}}\\\textbf{Mode}\\\end{tabular}}} & \multirow{2}{*}{\begin{tabular}[c]{@{}l@{}}\\\textbf{Description}\\\end{tabular}} & \multicolumn{2}{c|}{\textbf{Duration {[}min{]}}} \\ \cline{3-4}
\multicolumn{1}{|l|}{} &  & \multicolumn{1}{l|}{\begin{tabular}[c]{@{}c@{}}\textbf{Iani}\\\textbf{Chaos}\end{tabular}} & \multicolumn{1}{l|}{\begin{tabular}[c]{@{}c@{}}\textbf{Ismenius}\\\textbf{Cavus}\end{tabular}} \\ \hline
\multicolumn{1}{|l|}{\textbf{Idle - Day}} & \begin{tabular}[c]{@{}l@{}}Pre-Heating (2 hr).\\ UHF Communication (7 min).\\ Battery Charging.\end{tabular} & 88 & 88 \\ \hline
\multicolumn{1}{|l|}{\textbf{Idle - Night}} & \begin{tabular}[c]{@{}l@{}}UHF Communication (7 min).\\ Thermal Regulation.\end{tabular} & 88 & 88 \\ \hline
 & \multicolumn{1}{r|}{\textbf{Total Duration {[}min{]}}} & \textbf{999} & \textbf{999} \\ \cline{2-4}
\end{tabular}
\end{table}


\subsubsection{Hibernation Sol}
\label{sec:ReferenceSols:HibernationSol}
The Hibernation Sol is the rover's survival mode during a dust storm. In this setting the rover solely draws power from its battery hence only the heater and a timer are on in order to conserve energy.

\begin{table}[h]
\footnotesize
\centering
\caption{Reference Sol for hibernation.}
\label{tab:mission-scenario-hibernation-sol}
\begin{tabular}{|l|l|}
\hline
\textbf{Mode} & \textbf{Description} \\ \hline
\textbf{Hibernation - Day} & \begin{tabular}[c]{@{}l@{}}All day.\\ Battery draw kept to a minimum.\end{tabular} \\ \hline
\textbf{Hibernation - Night} & \begin{tabular}[c]{@{}l@{}}All night.\\ Battery draw kept to a minimum.\end{tabular} \\ \hline
\end{tabular}
\end{table}


\subsection{Summary}
\label{sec:ReferenceSols:SummaryAndConclusion}
The section presents selected rover's modes which were used to define reference Sols. Of particular interest is the \textit{Optimal Mode} mode which takes advantage of the rover's active suspect system to maximize power generation and battery charging during the \textit{Idle - Day} mode. Distinctions were made between two types of \textit{Traverse} modes with respect to flat or upslope terrain topography.

% Operation durations were established for each mode as a prerequesite to determining power budgets in the following chapter.


\clearpage
\section{Power Budget}
\label{sec:Design:PowerBudget}
This section details the power budget for the reference Sols used in mission planning. Propulsion power budgets for the rover's Traverse modes were determined from data collected during the SherpaTT field trials in Utah. Power budgets for other rover modes that make up the reference Sols were taken from \citeother{CDF2014}.

%The chapter is structured as follows: Propulsion power budget are presented in Section \ref{sec:PowerBudget:PropulsionPowerBudget} following an analysis of Mars analogue field test campaign data. Rover modes that were presented in the previous Chapter are complemented with power budgets in Section \ref{sec:PowerBudget:RoverModePowerBudget}. The chapter is then summarized in Section \ref{sec:PowerBudget:SummaryAndConclusion}.

\subsection{Propulsion Power}
\label{sec:PowerBudget:PropulsionPowerBudget}
SherpaTT's actively articulated suspension system consists of four wheeled-legs with a total of 20 motors. Each leg is equipped with three suspension motors and two drive motors. The suspension motors are responsible for Pan, \ac{IL}, and \ac{OL} revolute joint rotations whereas the drive motors are responsible for \ac{WS} and \ac{WD}. The distribution of these motors across each leg are shown in Figure \ref{fig:sherpatt-actively-articulated-suspension-system}. Propulsion power draw refers to the summation of suspension and drive motor power draws. These power draws have been studied in detail for SherpaTT during a Mars analogue field campaign in Utah \citeother{Cordes2018}.

\begin{figure}[h]
  \centering
  \hypersetup{linkcolor=captionTextColor}
  \includegraphics[width=0.6\linewidth]{sections/design/power-budget/images/sherpatt-actively-articulated-suspension-sytem.png}\\
  \caption[SherpaTT actively articulated suspension system]
          {SherpaTT actively articulated suspension system, taken from \citeother{Cordes2018}.}
  \label{fig:sherpatt-actively-articulated-suspension-system}
\end{figure}

Lack of motor optimisation as well as lower gravity and pressure on Mars permit the assumption that, given similar topology traversals, measured propulsion power draws are greater than those that would be observed on a Martian environment. This assumption is further supported when considering that SherpaTT's velocities during power draw measurements were much greater than what has been achieved on present and past Mars rover missions.

Available datasets from the Mars analogue field test campaign cover two flat surface runs and three steep upslope terrain runs. From the two upslope runs, the dataset with the worst-case maximum and mean propulsion power draw was used as the worst-case scenario. Hereafter, all mention of SherpaTT power draws will reference measurements included in these datasets. Measured power draws fluctuate due to slips, skids, noise, and other unknown imperfections. To ease readability, local minima, maxima, and media lines have been traced for all power plot figures.

\subsubsection{Flat Terrain Traverse}
\label{sec:PowerBudget:PropulsionPowerBudget:FlatTerrainTraverse}
Both \ac{MER} and \ac{MSL} rovers are each equipped with a total of 10 propulsion motors to drive their Rocker-Bogie passive suspension system: six to rotate the wheels and four to steer them \citeother{Novak2005} \citeother{Lakdawalla2018}. The \ac{MER} rovers needed approximately \SI{100}{\watt} to drive \citeother{MERRoverEnergy}. Equation \ref{eq:InitialPropulsionPowerEstimate} was used for to determine an initial SherpaTT propulsion power draw using a single \ac{MER} wheel power draw as an estimation unit:

\begin{align}
  \label{eq:InitialPropulsionPowerEstimate}
  P_{prop}^{sherpatt} &= \frac{P_{prop}^{mer}}{N_{wheels}^{mer}} \times N_{wheels}^{sherpatt} \times \left(1 +\frac{P_{susp}^{sherpatt}}{P_{prop}^{sherpatt}}\right) \\
           &= \frac{100}{6} \times 4 \times 1.17\\
           &= \SI{78}{\watt}
\end{align}


where $P_{prop}$ is the total propulsion power, $P_{susp}$ is the total suspension power, $N_{wheels}$ is the number of wheels, and $P_{susp}^{sherpatt} / P_{prop}^{sherpatt}$ is the suspension system's share of the total propulsion power. For the latter, a worst-case 17\% was taken from \citeother{Cordes2018} for data collected from a flat terrain outdoor setting.

The propulsion power draws measured for SherpaTT on flat surface runs are shown in Figure \ref{fig:plot:sherpatt-flat-terrain-power-draw}. These measurements are summarised in Table \ref{tab:sherpatt-flat-terrain-global-minimum-maximum-and-medium-power-draws}. To eliminate power draw fluctuations from the analysis, only local media values were considered. Local media were selected rather than the worst-case local maxima on the basis of the assumptions made in Section \ref{sec:PowerBudget:PropulsionPowerBudget}. For flat terrain traverses, a worst-case maximum power draw of \SI{74}{\watt} is observed in close accordance with the initial estimate obtained from Equation \ref{eq:InitialPropulsionPowerEstimate}.

\begin{figure}[h]
\captionsetup[subfigure]{justification=centering}
\vspace{-2ex}
	\centering
    %% setup sizes
    \setlength{\subfigureWidth}{0.50\textwidth}
    \setlength{\graphicsHeight}{80mm}
    %% kill hyper-link highlighting
    \hypersetup{hidelinks=true}%
    %% the figures
    \begin{subfigure}[t]{\subfigureWidth}
        \centering
        \includegraphics[height=\graphicsHeight]{sections/design/power-budget/plots/locomotion-power-draw-on-flat-terrain-1.png}
        \subcaption{Run \#1}
        \label{fig:plot:sub:sherpatt-flat-terrain-power-draw-1}
    \end{subfigure}\hfill
    \begin{subfigure}[t]{\subfigureWidth}
        \centering
        \includegraphics[height=\graphicsHeight]{sections/design/power-budget/plots/locomotion-power-draw-on-flat-terrain-2.png}
  		\subcaption{Run \#2}
		\label{fig:plot:sub:sherpatt-flat-terrain-power-draw-2}
	\end{subfigure}\\[0.8ex]
    \caption[Propulsion power draw for a flat terrain traverse during SherpaTT Mars analogue field tests in Utah]
            {Propulsion power draw for a flat terrain traverse during SherpaTT Mars analogue field tests in Utah.}
    \label{fig:plot:sherpatt-flat-terrain-power-draw}
\vspace{-2ex}
\end{figure}



\begin{table}[h]
\footnotesize
\centering
\caption[Global minimum, maximum, and medium of traced local minima, maxima, and media for SherpaTT flat terrain propulsion power draw lines]
    {Global minimum, maximum, and medium of traced local minima, maxima, and media for SherpaTT flat terrain propulsion power draw lines.}
\label{tab:sherpatt-flat-terrain-global-minimum-maximum-and-medium-power-draws}
\begin{tabular}{llccc}
\cline{3-5}
\multicolumn{2}{l|}{\multirow{2}{*}{}} & \multicolumn{3}{c|}{\textbf{Power Draw {[}W{]}}} \\ \cline{3-5}
\multicolumn{2}{l|}{} & \multicolumn{1}{c|}{\textbf{\begin{tabular}[c]{@{}c@{}}Global Minimum\end{tabular}}} & \multicolumn{1}{c|}{\textbf{\begin{tabular}[c]{@{}c@{}}Global Maximum\end{tabular}}} & \multicolumn{1}{c|}{\textbf{\begin{tabular}[c]{@{}c@{}}Global Media\end{tabular}}} \\ \hline
\multicolumn{1}{|c|}{\multirow{4}{*}{\textbf{Run \#1}}} & \multicolumn{1}{l|}{\textbf{Measured}} & \multicolumn{1}{c|}{27} & \multicolumn{1}{c|}{114} & \multicolumn{1}{c|}{60} \\ \cline{2-5}
\multicolumn{1}{|c|}{} & \multicolumn{1}{l|}{\textbf{Local Minima}} & \multicolumn{1}{c|}{27} & \multicolumn{1}{c|}{50} & \multicolumn{1}{c|}{38} \\ \cline{2-5}
\multicolumn{1}{|c|}{} & \multicolumn{1}{l|}{\textbf{Local Maxima}} & \multicolumn{1}{c|}{69} & \multicolumn{1}{c|}{114} & \multicolumn{1}{c|}{83} \\ \cline{2-5}
\multicolumn{1}{|c|}{} & \multicolumn{1}{l|}{\textbf{Local Media}} & \multicolumn{1}{c|}{52} & \multicolumn{1}{c|}{74} & \multicolumn{1}{c|}{61} \\ \hhline{|=|=|=|=|=|}
\multicolumn{1}{|l|}{\multirow{4}{*}{\textbf{Run \#2}}} & \multicolumn{1}{l|}{\textbf{Measured}} & \multicolumn{1}{c|}{17} & \multicolumn{1}{c|}{101} & \multicolumn{1}{c|}{51} \\ \cline{2-5}
\multicolumn{1}{|l|}{} & \multicolumn{1}{l|}{\textbf{Local Minima}} & \multicolumn{1}{c|}{17} & \multicolumn{1}{c|}{42} & \multicolumn{1}{c|}{30} \\ \cline{2-5}
\multicolumn{1}{|l|}{} & \multicolumn{1}{l|}{\textbf{Local Maxima}} & \multicolumn{1}{c|}{52} & \multicolumn{1}{c|}{101} & \multicolumn{1}{c|}{74} \\ \cline{2-5}
\multicolumn{1}{|l|}{} & \multicolumn{1}{l|}{\textbf{Local Media}} & \multicolumn{1}{c|}{40} & \multicolumn{1}{c|}{64} & \multicolumn{1}{c|}{52} \\ \hline
 &  & \multicolumn{1}{l}{} & \multicolumn{1}{l}{} & \multicolumn{1}{l}{} \\
 &  & \multicolumn{1}{l}{} & \multicolumn{1}{l}{} & \multicolumn{1}{l}{}
\end{tabular}
\end{table}


\pagebreak
\subsubsection{Upslope Terrain Traverse}
\label{sec:PowerBudget:PropulsionPowerBudget:UpslopeTerrainTraverse}
Propulsion power draws on a steep uplsope were measured along an approximately \SI{16}{\meter} track and are shown Figure \ref{fig:plot:sub:sherpatt-disaggregated-upslope-terrain-power-draw-locomotion}. An initial estimate of \SI{132}{\watt} was obtained using Equation \ref{eq:InitialPropulsionPowerEstimate} with a worst case 98\% suspension system's share of the total propulsion power taken from \citeother{Cordes2018} for data collected from a steep slope outdoor setting. The drive and suspension power draw components are shown in Figures \ref{fig:plot:sub:sherpatt-disaggregated-upslope-terrain-power-draw-drive} and \ref{fig:plot:sub:sherpatt-disaggregated-upslope-terrain-power-draw-suspension}, respectively.

\begin{figure}[h]
\captionsetup[subfigure]{justification=centering}
\vspace{-2ex}
	\centering
    %% setup sizes
    \setlength{\subfigureWidth}{0.32\textwidth}
    \setlength{\graphicsHeight}{50mm}
    %% kill hyper-link highlighting
    \hypersetup{hidelinks=true}%
    %% the figures
	\begin{subfigure}[t]{\subfigureWidth}
        \centering
        \includegraphics[height=\graphicsHeight]{sections/design/power-budget/plots/locomotion-power-draw-on-upslope-terrain.png}
  		\subcaption{Propulsion}
		\label{fig:plot:sub:sherpatt-disaggregated-upslope-terrain-power-draw-locomotion}
	\end{subfigure}\hfill
	\begin{subfigure}[t]{\subfigureWidth}
        \centering
        \includegraphics[height=\graphicsHeight]{sections/design/power-budget/plots/drive-power-draw-on-upslope-terrain.png}
  		\subcaption{Drive}
		\label{fig:plot:sub:sherpatt-disaggregated-upslope-terrain-power-draw-drive}
	\end{subfigure}\hfill
    \begin{subfigure}[t]{\subfigureWidth}
        \centering
        \includegraphics[height=\graphicsHeight]{sections/design/power-budget/plots/suspension-power-draw-on-upslope-terrain.png}
  		\subcaption{Suspension}
		\label{fig:plot:sub:sherpatt-disaggregated-upslope-terrain-power-draw-suspension}
	\end{subfigure}\\[0.8ex]
    \caption[Disaggregated measurements of power draw for upslope terrain traverse during SherpaTT Mars analogue field tests in Utah]
            {Disaggregated measurements of power draw for upslope terrain traverse during SherpaTT Mars analogue field tests in Utah.}
    \label{fig:plot:sherpatt-disaggregated-upslope-terrain-power-draw}
\vspace{-2ex}
\end{figure}

An uplsope traverse has no discernable effect on the suspension power draw, however; there is a clear gradual increase in the drive power draw. The global maximum, minimum, and medium of the traced local minima, maxima, and media power draw lines are presented in Table \ref{tab:sherpatt-upslope-terrain-global-minimum-maximum-and-medium-power-draws}.

\begin{table}[h]
\footnotesize
\centering
\caption{Global minimum, maximum, and medium of traced local minima, maxima, and media for SherpaTT upslope terrain traverse propulsion power draw lines.}
\label{tab:sherpatt-upslope-terrain-global-minimum-maximum-and-medium-power-draws}
\begin{tabular}{l|c|c|c|}
\cline{2-4}
\multicolumn{1}{c|}{\multirow{2}{*}{\textbf{}}} & \multicolumn{3}{c|}{\textbf{Power Draw {[}W{]}}} \\ \cline{2-4}
\multicolumn{1}{c|}{} & \textbf{\begin{tabular}[c]{@{}c@{}}Global Minimum\end{tabular}} & \textbf{\begin{tabular}[c]{@{}c@{}}Global Maximum\end{tabular}} & \textbf{\begin{tabular}[c]{@{}c@{}}Global Media\end{tabular}} \\ \hline
\multicolumn{1}{|l|}{\textbf{Measured}} & 34 & 218 & 114 \\ \hline
\multicolumn{1}{|l|}{\textbf{Local Minima}} & 34 & 172 & 98 \\ \hline
\multicolumn{1}{|l|}{\textbf{Local Maxima}} & 54 & 218 & 133 \\ \hline
\multicolumn{1}{|l|}{\textbf{Local Media}} & 40 & 188 & 18 \\ \hline
\end{tabular}
\end{table}


\pagebreak
Figure \ref{fig:plot:sherpatt-upslope-terrain-power-draw} overlaps the propulsion local media power draws with the tackled slope angles. The steepest slope angle was \SI{28}{\degree} for an average of \SI{17.52}{\degree}. Slope angle increase are consistently followed by power draw spikes, i.e. at approximately 3, 4, 5, 6, 8, and 9 \si{\meter} in the odometry measurements. Inversely, slope angle decreases were followed by power draws troughs at approximately 11, 13, 14, and 16 \si{\meter}.

\begin{figure}[h]
  \centering
  \hypersetup{linkcolor=captionTextColor}
  \includegraphics[width=0.8\linewidth]{sections/design/power-budget/plots/minima-locomotion-power-draws-on-upslope-terrain.png}\\
  \caption[Mean Propulsion power draw for an upslope terrain traverse during SherpaTT Utah field test campaign.]
          {Mean Propulsion power draw for an upslope terrain traverse during SherpaTT Utah field test campaign.}
  \label{fig:plot:sherpatt-upslope-terrain-power-draw}
\end{figure}


The power draws trough following the slope angle change from \SI{28}{\degree} to \SI{20}{\degree} at the \SI{11}{\meter} mark is subsequently followed by an unusual power draw increase and fluctuation. These measurements were discarded as they are outliers with respect to the power draw responses for the slope angle descreases that followed.

Table \ref{tab:sherpatt-upslope-terrain-local-media-measurement-summary} summarises the minimum, maximum, and mean local media propulsion power draws that were measured for different slope angles. Discarding the outlier measurements subsequent to the slope angle change from \SI{28}{\degree} to \SI{20}{\degree} at the \SI{11}{\meter} to \SI{13}{\meter} portion of the track, the maximum mean local media propulsion power draw is \SI{146}{\watt}, which is close to the initial estimate given by Equation \ref{eq:InitialPropulsionPowerEstimate}.

\begin{table}[h]
\centering
\caption{SherpaTT mean propulsion power draw measurements for different slope sections}
\label{tab:sherpatt-upslope-terrain-local-media-measurement-summary}
\begin{tabular}{cc|c|c|c|}
\cline{3-5}
\multicolumn{1}{l}{} & \multicolumn{1}{l|}{} & \multicolumn{3}{c|}{\textbf{Power {[}W{]}}} \\ \hline
\multicolumn{1}{|l|}{\textbf{Distance {[}m{]}}} & \multicolumn{1}{l|}{\textbf{Slope Angle {[}deg{]}}} & \multicolumn{1}{l|}{\textbf{Minimum}} & \multicolumn{1}{l|}{\textbf{Maximum}} & \multicolumn{1}{l|}{\textbf{Mean}} \\ \hline
\multicolumn{1}{|c|}{\textbf{1 $<$ x $\leq$ 3}} & 10 & 40 & 64 & 51 \\ \hline
\multicolumn{1}{|c|}{\textbf{3 $<$ x $\leq$ 4}} & 11 & 73 & 93 & 85 \\ \hline
\multicolumn{1}{|c|}{\textbf{4 $<$ x $\leq$ 5}} & 15 & 74 & 87 & 83 \\ \hline
\multicolumn{1}{|c|}{\textbf{5 $<$ x $\leq$ 6}} & 16 & 85 & 125 & 107 \\ \hline
\multicolumn{1}{|c|}{\textbf{6 $<$ x $\leq$ 7}} & 28 & 98 & 141 & 123 \\ \hline
\multicolumn{1}{|c|}{\textbf{7 $<$ x $\leq$ 8}} & 22 & 97 & 139 & 116 \\ \hline
\multicolumn{1}{|c|}{\textbf{8 $<$ x $\leq$ 9}} & 25 & 113 & 164 & 133 \\ \hline
\multicolumn{1}{|c|}{\textbf{9 $<$ x $\leq$ 11}} & 28 & 114 & 176 & 146 \\ \hline
\multicolumn{1}{|c|}{\textbf{11 $<$ x $\leq$ 13}} & 20 & 135 & 188 & 167 \\ \hline
\multicolumn{1}{|c|}{\textbf{13 $<$ x $\leq$ 14}} & 15 & 119 & 123 & 145 \\ \hline
\multicolumn{1}{|c|}{\textbf{14 $<$ x $\leq$ 16}} & 10 & 70 & 186 & 94 \\ \hline
\end{tabular}
\end{table}


\pagebreak
\subsection{Traverse Power Budget}
\label{sec:PowerBudget:PowerBudget:TraversePowerBudget}
Worst-case daily insolations for an optical depth of $\tau = 1$ were used to determine the energy requirements of the flat and upslope traverse reference Sols and were taken from Tables \ref{tab:insolation-iani-chaos-clear-and-dusty-days} for Iani Chaos and \ref{tab:insolation-ismenius-cavus-clear-and-dusty-days} for Ismenius Cavus. The minimum required traverse distance was assumed to be \SI{5}{\meter} at both mission sites.

\subsubsection{Flat Traverse}
\label{sec:Design:PowerBudget:TraversePowerBudget:FlatTraverse}

Power and duration of \textit{DTE Communication}, \textit{Science Stop - Short}, and \textit{Hibernation} modes were taken from \citeother{CDF2014} and assigned to the reference Sols presented in Section \ref{sec:ReferenceSols:ReferenceSols}. Power draw estimates made in Sections \ref{sec:PowerBudget:PropulsionPowerBudget:FlatTerrainTraverse} and \ref{sec:PowerBudget:PropulsionPowerBudget:UpslopeTerrainTraverse} were used for \textit{Traverse} modes. Power draw for the \textit{Optimal Pose} mode was equated to that of the rover's flat terrain propulsion power and etimated to take up to \SI{10}{\minute}. The worst-case power budget for the reference \textit{Flat Traverse Sol} is shown in Table \ref{tab:worst-case-traverse-sol-power-budget}.

\begin{table}[h]
\footnotesize
\centering
\caption{Worst-case mission site flat terrain traverse Sol power budget for $\tau=1$.}
\label{tab:worst-case-traverse-sol-power-budget}
\begin{tabular}{lc|c|c|c|c|}
\cline{3-6}
 & \textbf{} & \multicolumn{2}{c|}{\textbf{\begin{tabular}[c]{@{}c@{}}Iani Chaos\\ $Ls=\SI{81}{\degree}$\end{tabular}}} & \multicolumn{2}{c|}{\textbf{\begin{tabular}[c]{@{}c@{}}Ismenius Cavus\\ $Ls=\SI{273}{\degree}$\end{tabular}}} \\ \hline
\multicolumn{1}{|l|}{\textbf{Mode}} & \textbf{P {[}W{]}} & \textbf{t {[}min{]}} & \textbf{E {[}Wh{]}} & \textbf{t {[}min{]}} & \textbf{E {[}Wh{]}} \\ \hline
\multicolumn{1}{|l|}{\textbf{Idle - Day}} & 29 & 603 & 291 & 464 & 224 \\ \hline
\multicolumn{1}{|l|}{\textbf{DTE Communication}} & 52 & 35 & 30 & 35 & 30 \\ \hline
\multicolumn{1}{|l|}{\textbf{Traverse - Flat}} & 113\footnote{Power draws taken from \citeother{CDF2014} for Communications, \ac{DHS}, \ac{GNC}, and \ac{PCDU} are added to the \SI{75}{\watt} flat terrain propulsion power draw resulting in a total \textit{Traverse Mode} power budget of \SI{113}{\watt}.} & 4.8 & 9 & 4.8 & 9 \\ \hline
\multicolumn{1}{|l|}{\textbf{Science Stop - Short}} & 60 & 60 & 60 & 60 & 60 \\ \hline
\multicolumn{1}{|l|}{\textbf{Optimal Pose}} & 75 & 12 & 13 & 10 & 13 \\ \hline
\multicolumn{1}{|l|}{\textbf{Idle - Night}} & 20 & 242 & 242 & 866 & 289 \\ \hline
\multicolumn{1}{|r|}{\textbf{Total}} & \textbf{349} & \textbf{1440} & \textbf{646} & \textbf{1440} & \textbf{625} \\ \hline
\multicolumn{1}{|r|}{\textbf{Total +20\% System Margin}} & \textbf{419} & - & \textbf{775} & - & \textbf{750} \\ \hline
\end{tabular}
\end{table}


The total energy requirements for these worst-case reference Sols were used as the minimum energy output when sizing the \ac{SA}. A 20\% system margin was applied to account for unaccounted inefficiencies such as \ac{PCDU} losses.

\subsubsection{Upslope Traverse}
\label{sec:Design:PowerBudget:TraversePowerBudget:UpslopeTraverse}
The worst-case upslope traverese reference Sols shown in Table \ref{tab:worst-case-upslope-traverse-sol-power-budget} only differ from their flat traverse equivalents by the \textit{Traverse} mode power draw. It is increased from \SI{75}{\watt} to \SI{150}{\watt}.

\begin{table}[h]
\footnotesize
\centering
\caption[Worst case upslope traverse Sol power budget at mission sites]
    {Worst case upslope traverse Sol power budget at mission sites for $\tau =1$.}
\label{tab:worst-case-upslope-traverse-sol-power-budget}
\begin{tabular}{lc|c|c|c|c|}
\cline{3-6}
 & \textbf{} & \multicolumn{2}{c|}{\textbf{\begin{tabular}[c]{@{}c@{}}Iani Chaos\\ $Ls=\SI{81}{\degree}$\end{tabular}}} & \multicolumn{2}{c|}{\textbf{\begin{tabular}[c]{@{}c@{}}Ismenius Cavus\\ $Ls=\SI{273}{\degree}$\end{tabular}}} \\ \hline
\multicolumn{1}{|l|}{\textbf{Mode}} & \textbf{P {[}W{]}} & \textbf{t {[}min{]}} & \textbf{E {[}Wh{]}} & \textbf{t {[}min{]}} & \textbf{E {[}Wh{]}} \\ \hline
\multicolumn{1}{|l|}{\textbf{Idle - Day}} & 29 & 603 & 291 & 464 & 224 \\ \hline
\multicolumn{1}{|l|}{\textbf{DTE Communication}} & 52 & 35 & 30 & 35 & 30 \\ \hline
\multicolumn{1}{|l|}{\textbf{Traverse - Upslope}} & 188\footnotemark & 4.8 & 9 & 4.8 & 15 \\ \hline
\multicolumn{1}{|l|}{\textbf{Science Stop - Short}} & 60 & 60 & 60 & 60 & 60 \\ \hline
\multicolumn{1}{|l|}{\textbf{Optimal Pose}} & 75 & 12 & 13 & 10 & 13 \\ \hline
\multicolumn{1}{|l|}{\textbf{Idle - Night}} & 20 & 242 & 242 & 866 & 289 \\ \hline
\multicolumn{1}{|r|}{\textbf{Total}} & \textbf{424} & \textbf{1440} & \textbf{652} & \textbf{1440} & \textbf{631} \\ \hline
\multicolumn{1}{|r|}{\textbf{Total +20\% System Margin}} & \textbf{509} & - & \textbf{782} & - & \textbf{757} \\ \hline
\end{tabular}
\end{table}
\footnotetext{The \SI{188}{\watt} \textit{Traverse Mode} power draw is obtained by adding Communication, \ac{DHS}, \ac{GNC}, and \ac{PCDU} power draws to the \SI{150}{\watt} flat terrain propulsion power draw. These additional power draws are taken from \citeother{CDF2014}.}


These worst-case reference Sols were not considered in \ac{SA} sizing. Regardless, they are presented for the sake of comparative analysis with the flat traverse equivalent.


\subsubsection{Hibernation}
\label{sec:Design:PowerBudget:TraversePowerBudget:Hibernation}
The power budget for the reference \textit{Hibernation Sol} is shown in Table \ref{tab:hibernation-sol-power-budget}. The \SI{18}{\watt} power draw was taken from \citeother{CDF2014}. No system margins are applied for this reference Sol.

\begin{table}[h]
\footnotesize
\centering
\caption{Hibernation Sol power budget}
\label{tab:hibernation-sol-power-budget}
\begin{tabular}{lc|c|c|c|c|}
\cline{3-6}
 & \textbf{} & \multicolumn{2}{c|}{\textbf{Iani Chaos}} & \multicolumn{2}{c|}{\textbf{Ismenius Cavus}} \\ \hline
\multicolumn{1}{|l|}{\textbf{Mode}} & \textbf{P {[}W{]}} & \textbf{t {[}min{]}} & \textbf{E {[}Wh{]}} & \textbf{t {[}min{]}} & \textbf{E {[}Wh{]}} \\ \hline
\multicolumn{1}{|l|}{\textbf{Hibernation - Day}} & 18 & 720 & 216 & 720 & 216 \\ \hline
\multicolumn{1}{|l|}{\textbf{Hibernation - Night}} & 18 & 720 & 216 & 720 & 216 \\ \hline
\multicolumn{1}{|r|}{\textbf{Total}} & \textbf{36} & \textbf{1440} & \textbf{432} & \textbf{1440} & \textbf{432} \\ \hline
\multicolumn{1}{|r|}{\textbf{Total +20\% System Margin}} & \textbf{43} & - & \textbf{518} & - & \textbf{518} \\ \hline
\end{tabular}
\end{table}


\subsection{Summary}
\label{sec:PowerBudget:Summary}
\todo[inline]{\textbf{TODO:} Write section summary.}


\clearpage
\section{Requirements and Design Drivers}
\label{sec:Design:RequirementsAndDesignDrivers}
Requirements and design drivers constrain the rover's \ac{SA} design. This section reviews and complements the assumptions made thus far. The selected solar cell technology and the assumptions pertaining to its \ac{BOL} and \ac{EOL} efficiencies are also presented. Further assumptions are made regarding the rover's mobility capabilities.


\subsection{Assumptions}
\label{sec:RequirementsAndDesignDrivers:Assumptions}
The assumptions are:

\begin{enumerate}[label=\textbf{\textcolor{BulletBlue}{A-\arabic*}}]
    \item Propulsion power draw for flat terrain traverses is \SI{75}{\watt}.
    \item Propulsion power draw for upslope terrain traverses up to \SI{30}{\degree} inclination is \SI{150}{\watt}.
    \item Solar cell technology is \ac{IMM} solar cell with AM0 efficiency of 32.3\% at \ac{BOL} and 22.5\% at \ac{EOL} \citepower{Sharps2017}.
    \item \label{itm:ass:solar_cell_efficiency} \ac{IMM} Solar cell efficiency on Mars surface is 22\% at \ac{EOL}.
    \item \label{itm:ass:packing_efficiency} Solar cell packing efficiency is 85\%.
    \item \label{itm:ass:red_shifts} Solar array efficiency varies by 3\% due to changing temperature and red-shift spectral losses through the day time-period \citepower{Kerslake1999}.
    \item \label{itm:ass:dust_deposition_saturation} Solar array dust performance degradation saturates at 30\% \citepower{Stella2005}.
    \item \label{itm:ass:protruding_shadowing} Solar array performance degradation from shadowing by protruding rover structures is 5\%.
    \item \label{itm:ass:sa_surface_density} The \ac{SA} surface density is \SI{3.7}{kg.m^{-2}}.
    \item The rover's nominal velocity is at least \SI{2}{cm.s^{-2}}.
    \item There is an average 15\% slippage when traversing a flat terrains for any given distance \citeother{Cordes2018}.
    \item There is an average 30\% slippage when traversing an upslope terrain for any given distance and slope angle \citeother{Cordes2018}.
\end{enumerate}

\subsection{Requirements}
\label{sec:sec:Design:RequirementsAndDesignDrivers:Requirements}
The following considerations are taken into account to narrow down the requirements:

\begin{enumerate}[label=\textbf{\textcolor{BulletBlue}{(\alph*)}}]
    \item Interest in long traverses from the PERASPERA programme on space robotics \citeother{PERASPERA}.
    \item Focus on space robotics development for long autonomous traverses at \ac{DFKI} \citeother{OG6} \citeother{OG10}.
    \item Mission site at Iani Chaos is approximately \SI{100}{\kilo\meter} from closest resource deposits of interest.
    \item Area of interest at Ismenius Cavus is approximately $\SI{100}{\kilo\meter} \times \SI{50}{\kilo\meter}$.
    \item Dust storm tracking in \citemarsenv{Battalio2019} observed that the average duration of local dust storms is approximately seven Sols.
\end{enumerate}

The following \ac{EOL} requirements apply for a mission life time of one \ac{MY}:

\begin{enumerate}[label=\textbf{\textcolor{BulletBlue}{R-\arabic*}}]
    %\item The rover shall be able to traverse flat terrain at an optical depth of $\tau = 1$.
    %\item The rover shall be able to traverse \SI{30}{\degree} inclined terrain at an optical depth of $\tau = 1$.
    \item \label{itm:req:total_distance_flat_terrain} The rover shall traverse flat terrain for a total distance of at least \SI{10}{\kilo\meter}.
    \item \label{itm:req:survive_tau1} The rover shall survive optical depths of up to $\tau = 1$.
    \item \label{itm:req:survice_tau2} The rover shall survive optical depths $\tau = 2$ for at least seven Sols.
\end{enumerate}

\subsection{Constraint}
\label{sec:Design:RequirementsAndDesignDrivers:Constraints}
The design must allow for \textit{non-Traverse Sols} such as long science stops, intermittent hibernation periods during dust storms, battery recharging periods, or limited activities during a Solar Conjunction Break. Furthermore, traverses may only occur during daylight to allow terrain observations to be made in case of slip and skids that pass a yet to be defined threshold. The scope of this thesis consider the following constraints:

\begin{enumerate}[label=\textbf{\textcolor{BulletBlue}{C-\arabic*}}]
    \item There shall be a minimum of at least two \textit{non-Traverse Sols} between each \textit{Traverse Sol}.
    \item \label{itm:con:daylight_traverse} The rover can be in \textit{Traverse mode} only during daylight.
\end{enumerate}

\subsection{Design Drivers}
\label{sec:Design:RequirementsAndDesignDrivers:DesignDrivers}
The \ac{SA} design drivers are:

\begin{enumerate}[leftmargin=1.31cm,label=\textbf{\textcolor{BulletBlue}{DD-\arabic*}}]
    \item \label{itm:dd:limited_unfolding} Limited amount of unfolding during deployment.
    \item \label{itm:dd:shadowing} Shadowing is minimized.
    \item \label{itm:dd:end_effector} Manipulator arm end effector can access the ground.
    \item \label{itm:dd:four_plis} Manipulator arm can access at least four \ac{PLI} stored in the rover.
    \item \label{itm:dd:unobstructed} Movement range of the suspension system is unobstructed.
    \item \label{itm:dd:cog} Position of the rover's \ac{CoG} along the x and y axis is centered during stowed position.
\end{enumerate}

%\subsection{Summary}
%\label{sec:Design:RequirementsAndDesignDrivers:Summary}
%The assumptions, requirements, constraints, and design drivers presented in this section apply to both missions sites and are used to size the rover's \acp{SA}.


\clearpage
\section{Solar Array}
\label{sec:Design:SolarArray}
\todo[inline]{\textbf{TODO:} Write section introduction.}

\subsection{Sizing}
\todo[inline]{\textbf{TODO:} Short introduction.}

\subsubsection{Solar Array}
Equation \ref{eq:SA_slope_energy} is rearranged in into an expression of the solar cell coverage area $A$, shown in Equation \ref{eq:solar_cell_coverage_area}:

\begin{equation}
  \label{eq:solar_cell_coverage_area}
  A = \frac{E}{\eta \cdot H_{\beta} \cdot PR}
\end{equation}

where $E$ becomes the energy required by the rover on that Sol and $H_{h}$ the worst-case available daily insolation, henceforth $E_{req}^{worst}$ and $H_{\beta}^{worst}$ respectively. $H_{\beta}^{worst}$ is used instead of $H_{h}^{worst}$ in order to take advantage of the rover's active suspension system to generate the most available worst case daily insolation. The values for these varibles are taken from assumptions and previously calculated energies and daily insolations:

\begin{enumerate}[label=\textbf{\textcolor{BulletBlue}{(\alph*)}}]
    \item From Table \ref{tab:worst-case-traverse-sol-power-budget}, $E_{req}^{worst}$ is \SI{775}{\watt\hour} at Iani Chaos and \SI{750}{\watt\hour} at Ismenius Cavus.
    \item From Tables \ref{tab:insolation-iani-chaos-clear-and-dusty-days} and \ref{tab:insolation-ismenius-cavus-clear-and-dusty-days}, $H_{\beta}^{worst}$ for $\tau=1$ is \SI{2479}{Wh.m^{-2}} at Iani Chaos and \SI{1345}{Wh.m^{-2}} at Ismenius Cavus.
    \item From \ref{itm:ass:red_shifts}, \ref{itm:ass:dust_deposition_saturation}, and \ref{itm:ass:protruding_shadowing}, \ac{PR} at \ac{EOL} is $PR_{EOL} = 1 - (0.03 + 0.3 + 0.05) = 0.62$.
    \item From \ref{itm:ass:solar_cell_efficiency}, $\eta_{EOL} = 0.22$.
\end{enumerate}


The required solar cell coverage area at Iani Chaos was thus determined:
\begin{align}
  \label{calc:solar_cell_area_iani_chaos}
  A_{iani} &= \frac{E_{req}^{worst}}{\eta_{EOL} \cdot H_{\beta}^{worst} \cdot PR_{EOL}}\\
           &= \frac{775}{0.22 \cdot 2479 \cdot 0.62}\\
           &= \SI{2.29}{m^{2}}
\end{align}

and at Ismenius Cavus:
\begin{align}
  \label{calc:solar_cell_area_ismenius_cavus}
  A_{ismenius} &= \frac{E_{req}^{worst}}{\eta_{EOL} \cdot H_{\beta}^{worst} \cdot PR_{EOL}}\\
               &= \frac{750}{0.22 \cdot 1345 \cdot 0.62}\\
               &= \SI{4.09}{m^{2}}
\end{align}

At Iani Chaos, from \ref{itm:ass:packing_efficiency} the resulting \ac{SA} area was \SI{2.7}{m^{2}} and from \ref{itm:ass:sa_surface_density} its mass was 9.95 \si{\kilo\gram}. Taking advantage of \ac{SA} inclination capabilities with $\beta_{best} = \SI{10}{\degree}$ resulted in \ac{SA} sizing decrease of \SI{3.9}{\percent} when compared with a horizontal surface configuration. For $\tau = 1$ during global dust storm season and $\tau = 0.4$ for the remainder of the year, the total maximum flat traverse distance achievable over the course of one \ac{MY} was increased by \SI{8.79}{\percent} from \SI{59.13}{\kilo\meter} to \SI{64.33}{\kilo\meter}.

At Ismenius Cavus, the resulting \ac{SA} area and mass were \SI{4.8}{m^{2}} at 17.8 \si{\kilo\gram}. The \ac{SA} area was decreased by \SI{4.6}{\percent}. The total maximum achievable flat traverse distance achievable during one \ac{MY} was increased by \SI{1.39}{\percent} from \SI{67.4}{\kilo\meter} to \SI{68.34}{\kilo\meter}.

The traverse distance gains attributed to \ac{SA} inclination capabilities did not seem to justify adopting the complexities of an active suspension system for the purpose of increasing traverse distance via solar tracking. This was particularly true at Ismenius Cavus. The savings in \ac{SA} surface area and mass also left much to be desired. The size problem of the resulting \ac{SA} areas are illustrated in Figure \ref{fig:sa-area-initial-sizes}.

\clearpage
\begin{figure}[h]
  \centering
  \hypersetup{linkcolor=captionTextColor}
  \includegraphics[width=0.8\linewidth]{sections/design/solar-array/images/sa-area-initial-sizes.png}\\
  \caption[Size comparison between SherpaTT and initial SA areas for Iani Chaos and Ismenius Cavus]
          {Size comparison between SherpaTT and initial \ac{SA} areas for Iani Chaos and Ismenius Cavus. The outlined square areas are equivalent to \ac{SA} areas of \SI{2.7}{m^{2}} for Iani Chaos and \SI{4.8}{m^{2}} for Ismenius Cavus.}
  \label{fig:sa-area-initial-sizes}
\end{figure}

To explain the lack of significant gain with $\beta_{best}$, the generated \ac{SA} energy and maximum traverse durations are plotted in Figures \ref{fig:plot:iani-chaos-generated-energy-and-max-traverse-durations} for Iani Chaos and Figure \ref{fig:plot:ismenius-cavus-generated-energy-and-max-traverse-durations} for Ismenius Cavus.

The \ref{itm:con:daylight_traverse} constraint imposed an upper limit to the maximum traverse durations. This ceiling corresponded to the daylight time that was available for traversing. At Iani Chaos, this resulted in the \ac{SA} generating excess propulsion energy which could not be used. For the scenario plotted in Figure \ref{fig:plot:sub:ismenius-cavus-max-traverse-durations}, this occured before and after the global storm season where both horizontal and $\beta_{best}$ inclined \ac{SA} surfaces generated excess energy which annulled any gains associated with solar tracking. This would have also have been the case during the global storm season for clear days with $\tau = 0.4$.

At Ismenius Cavus, the excess energy problem was further pronounced than in Iani Chaos. As seen in the scenario plotted in Figure \ref{fig:plot:sub:iani-chaos-max-traverse-durations}, an inclined \ac{SA} surface was only relevant during the global dust storm season with a dusty atmosphere of $\tau = 1$. This negligeable advantage resulted in the limited \SI{1.39}{\percent} traverse distance gain with $\beta_{best}$ that was noted earlier.

The \ac{SA} sizing approach using Equations \ref{calc:solar_cell_area_iani_chaos} and \ref{calc:solar_cell_area_ismenius_cavus} had to be re-evaluated towards a solution that resulted in significant gains with $\beta_{best}$, bot in terms of \ac{SA} area reduction and in attainable traverse distances when compared to hose obtained from a horizontal \ac{SA} surface.


\begin{figure}[h]
\captionsetup[subfigure]{justification=centering}
\vspace{-2ex}
	\centering
    %% setup sizes
    \setlength{\subfigureWidth}{0.50\textwidth}
    \setlength{\graphicsHeight}{80mm}
    %% kill hyper-link highlighting
    \hypersetup{hidelinks=true}%
    %% the figures
    \begin{subfigure}[t]{\subfigureWidth}
        \centering
        \includegraphics[height=\graphicsHeight]{sections/design/solar-array/plots/ianichaos-daily-generated-energy.png}
        \subcaption{Generated Energy}
        \label{fig:plot:sub:iani-chaos-generated-energy}
    \end{subfigure}\hfill
    \begin{subfigure}[t]{\subfigureWidth}
        \centering
        \includegraphics[height=\graphicsHeight]{sections/design/solar-array/plots/ianichaos-75w-max-traverse-durations.png}
  		\subcaption{Maximum Traverse Durations}
		\label{fig:plot:sub:iani-chaos-max-traverse-durations}
	\end{subfigure}\\[0.8ex]
    \caption[Generated energy and maxium achievable flat terrain traverse durations at Iani Chaos]
            {Generated energy and maxium achievable flat terrain traverse duration at Iani Chaos. Optical depth  $\tau = 1$ was used for global dust storm season ($\SI{185}{\degree} \leq L_{s} \leq \SI{315}{\degree}$) and $\tau = 0.4$ for the remainder of the year. The \textit{available daylight traverse time} corresponds to the amount of daylight hours left in a \textit{Traverse Sol} after subtracting the time taken by non-Traverse modes: \textit{Idle - Day}, \textit{\ac{DTE} Communication}, \textit{Science Stop - Short}, and \textit{Optimal Pose}. The maximum traverse durations for \ac{SA} horizontal do not consider the \textit{Optimal Pose} mode.}
    \label{fig:plot:iani-chaos-generated-energy-and-max-traverse-durations}
\vspace{-2ex}
\end{figure}

\begin{figure}[h]
\captionsetup[subfigure]{justification=centering}
\vspace{-2ex}
	\centering
    %% setup sizes
    \setlength{\subfigureWidth}{0.50\textwidth}
    \setlength{\graphicsHeight}{80mm}
    %% kill hyper-link highlighting
    \hypersetup{hidelinks=true}%
    %% the figures
    \begin{subfigure}[t]{\subfigureWidth}
        \centering
        \includegraphics[height=\graphicsHeight]{sections/design/solar-array/plots/ismeniuscavus-daily-generated-energy.png}
        \subcaption{Generated Energy}
        \label{fig:plot:sub:ismenius-cavus-generated-energy}
    \end{subfigure}\hfill
    \begin{subfigure}[t]{\subfigureWidth}
        \centering
        \includegraphics[height=\graphicsHeight]{sections/design/solar-array/plots/ismeniuscavus-75w-max-traverse-durations.png}
  		\subcaption{Maximum Traverse Durations}
		\label{fig:plot:sub:ismenius-cavus-max-traverse-durations}
	\end{subfigure}\\[0.8ex]
    \caption[Generated energy and maxium achievable flat terrain traverse durations at Ismenius Cavus]
            {Generated energy and maxium achievable flat terrain traverse duration at Ismenius Cavus. The same considerations were taken as in Figure \ref{fig:plot:iani-chaos-generated-energy-and-max-traverse-durations}.}
    \label{fig:plot:ismenius-cavus-generated-energy-and-max-traverse-durations}
\vspace{-2ex}
\end{figure}

\begin{figure}[h]
\captionsetup[subfigure]{justification=centering}
\vspace{-2ex}
	\centering
    %% setup sizes
    \setlength{\subfigureWidth}{0.50\textwidth}
    \setlength{\graphicsHeight}{80mm}
    %% kill hyper-link highlighting
    \hypersetup{hidelinks=true}%
    %% the figures
    \begin{subfigure}[t]{\subfigureWidth}
        \centering
        \includegraphics[height=\graphicsHeight]{sections/design/solar-array/plots/ianichaos-75w-traverse-gains-for-27m2-sa-area.png}
        \subcaption{\ac{SA} area = \SI{2.7}{\meter\squared}}
        \label{fig:plot:sub:iani-chaos-flat-traverse-gains-for-initial-sa-area}
    \end{subfigure}\hfill
    \begin{subfigure}[t]{\subfigureWidth}
        \centering
        \includegraphics[height=\graphicsHeight]{sections/design/solar-array/plots/ianichaos-75w-traverse-gains-for-different-sa-areas.png}
  		\subcaption{For different \ac{SA} areas}
		\label{fig:plot:sub:iani-chaos-flat-traverse-gains-for-different-sa-area}
	\end{subfigure}\\[0.8ex]
    \caption[Flat traverse distance gains at Iani Chaos]
            {Flat traverse distance gains at Iani Chaos.}
    \label{fig:plot:iani-chaos-flat-traverse-gains}
\vspace{-2ex}
\end{figure}




\begin{figure}[h]
\captionsetup[subfigure]{justification=centering}
\vspace{-2ex}
	\centering
    %% setup sizes
    \setlength{\subfigureWidth}{0.50\textwidth}
    \setlength{\graphicsHeight}{80mm}
    %% kill hyper-link highlighting
    \hypersetup{hidelinks=true}%
    %% the figures
    \begin{subfigure}[t]{\subfigureWidth}
        \centering
        \includegraphics[height=\graphicsHeight]{sections/design/solar-array/plots/ismeniuscavus-75w-traverse-gains-for-48m2-sa-area.png}
        \subcaption{\ac{SA} area = \SI{4.8}{\meter\squared}}
        \label{fig:plot:sub:ismenius-chaos-flat-traverse-gains-for-initial-sa-area}
    \end{subfigure}\hfill
    \begin{subfigure}[t]{\subfigureWidth}
        \centering
        \includegraphics[height=\graphicsHeight]{sections/design/solar-array/plots/ismeniuscavus-75w-traverse-gains-for-different-sa-areas.png}
  		\subcaption{For different \ac{SA} areas}
		\label{fig:plot:sub:ismenius-chaos-flat-traverse-gains-for-different-sa-area}
	\end{subfigure}\\[0.8ex]
    \caption[Flat traverse distance gains at Ismnenius Chaos]
            {Flat traverse distance gains at Ismnenius Chaos.}
    \label{fig:plot:ismenius-chaos-flat-traverse-gains}
\vspace{-2ex}
\end{figure}


\todo[inline]{\textbf{TODO:} Size SA with hibernation power budget. Reduce Hibernation mode power draw.}

%\label{itm:req:total_distance_flat_terrain}

\clearpage
\subsubsection{Battery}
\todo[inline]{\textbf{TODO:} Battery size based on energy required to keep the rover Warm through the night. Check if the calculated size satisfies the hibernation requirement. If not, resize.}

\subsection{Baseline Design}
\todo[inline]{\textbf{TODO:} Refer to design drivers and present solution.}

\subsection{Mechanisms}
\todo[inline]{\textbf{TODO:} Short introduction.}

\subsubsection{Deployment}
\todo[inline]{\textbf{TODO:} 1. Sequence and 2. Cantilever beam analysis at every deployment step.}

\subsubsection{HDRMs}
\todo[inline]{\textbf{TODO:} Calculate for Force acting on the \ac{HDRM}.}

\subsection{Summary}
\todo[inline]{\textbf{TODO:} Write section summary.}


\section{Simulation}
\label{sec:Design:Simulation}
The proposed rover redesigns and \ac{SA} configurations for both mission sites are modeled with Blender/Phobos. Phobos is ``an add-on for the open-source 3D modeling software Blender that enables the creation of robot models for use in robot frameworks like ROS and ROCK or in real-time simulations such as MARS'' \citeother{Phobos}. MARS is ``a cross-platform simulation and visualisation tool created for robotics research. It consists of a core framework containing all main simulation components, a GUI (based on Qt), 3D visualization (using OSG) and a physics engine (based on ODE)'' \citeother{MARSSim}. \refFig{fig:simulated-mission-site-ismenius-cavus} shows the robot model loaded on the MARS platform to simulatie mission scenarios using a \ac{HiRISE} \ac{DTM} of a well preserved crater at Ismenius Cavus.  The simulated solar power output data is produced by a \ac{PMS} implemented as part of this thesis which is integrated with the already available robot simulation toolkit.

\begin{figure}[h]
  \captionsetup[subfigure]{justification=centering}
  \centering
  \hypersetup{linkcolor=captionTextColor}
  \includegraphics[width=1\linewidth]{sections/design/simulation/images/mars-sim-ismenius-cavus.png}\\
  \caption[Simulation of the rover inside a well preserved crater at Ismenius Cavus]
          {Simulation of the rover inside a well preserved crater at Ismenius Cavus.}
  \label{fig:simulated-mission-site-ismenius-cavus}
\end{figure}

\subsection{Z-Axis Revolutions}

\refFig{fig:simulation-data-rover-revolution-generated-power} and \refFig{fig:simulation-data-rover-revolution-generated-power-polar} are two different representations of the same solar generated power data obtained from commanding the rover to execute a \SI{10}{\degree} forward body-pitch followed by several revolutions around its z-axis. These revolutions result in a sinusoidal generated power variation.

\begin{figure}[h]
  \captionsetup[subfigure]{justification=centering}
  \centering
  \hypersetup{linkcolor=captionTextColor}
  \includegraphics[width=0.5\linewidth]{sections/design/simulation/plots/rover-generated-power.png}\\
  \caption[Power generated by the rover's solar array for multiple revolutions along its z-axis]
          {Power generated by the rover's solar array for multiple revolutions along its z-axis. Simulation conditions is solar noon at Ismenius Cavus.}
  \label{fig:simulation-data-rover-revolution-generated-power}
\end{figure}

In \refFig{fig:simulation-data-rover-revolution-generated-power-polar}, the angle values represent the direction faced by the inclined solar array. Maximum power is generated when the rover's solar panels are facing South towards the equator. Inversely, facing northwards away from the equator genrates the least amount of power.


\begin{figure}[h]
  \captionsetup[subfigure]{justification=centering}
  \centering
  \hypersetup{linkcolor=captionTextColor}
  \includegraphics[width=0.4\linewidth]{sections/design/simulation/plots/rover-revolution-generated-power.png}\\
  \caption[Polar visualization of power generated by the rover's solar array for multiple revolutions along its z-axis]
          {Polar visualization of power generated by the rover's solar array for multiple revolutions along its z-axis. Angle values represent the direction faced by the inclined solar array. \SI{0}{\degree} is South, \SI{90}{\degree} is East, \SI{180}{\degree} is North, and \SI{270}{\degree} is West. Simulation conditions is solar noon at Ismenius Cavus.}
  \label{fig:simulation-data-rover-revolution-generated-power-polar}
\end{figure}

\clearpage
\subsection{Slope Compensation}

\refFig{fig:rover-counter-slope} shows a simulated scenario in which the rover is on a \SI{30}{\degree} inclined slope. The slope's inclination is facing opposite the equator resulting in a worst case power generation. The rover is positioned with a backward body-pitch of \SI{10}{\degree} in order to counter the slope induced forward inclination thus reducing \ac{SA} inclination angle from $\beta=\SI{30}{\degree}$ to $\beta=\SI{20}{\degree}$.

\begin{figure}[h]
  \captionsetup[subfigure]{justification=centering}
  \centering
  \hypersetup{linkcolor=captionTextColor}
  \includegraphics[width=0.4\linewidth]{sections/design/simulation/images/counter-slope.png}\\
  \caption[Simulation of the rover on an inclined slope.]
          {Simulation of the rover on an inclined slope.}
  \label{fig:rover-counter-slope}
\end{figure}

\refFig{fig:rover-counter-slope-measurements} shows solar power outputs for different body-pitch configuration while the rover is on a \SI{30}{\degree} slope. The initial state of the body-pitch is \SI{0}{\degree} follow by two seperate \SI{5}{\degree} forward pitch increments which worsen solar power generation by changing $\beta=\SI{30}{\degree}$ to $\beta=\SI{40}{\degree}$. After which, four separate \SI{5}{\degree} backward pitch decrements are executed, progressively improving power generation as $\beta$ is changed from \SI{40}{\degree} to \SI{20}{\degree}. As a final task, the rover is commanded to drive forward and it encounters uneven terrain which introduces minor fluctuations to the solar power output.

\vspace{0.5cm}

\begin{figure}[h]
  \captionsetup[subfigure]{justification=centering}
  \centering
  \hypersetup{linkcolor=captionTextColor}
  \includegraphics[width=0.9\linewidth]{sections/design/simulation/plots/counter-slope-plot.png}\\
  \caption[Solar power generated on sloped terrain with different SA inclinations.]
          {Solar power generated on sloped terrain with different SA inclinations.}
  \label{fig:rover-counter-slope-measurements}
\end{figure}


\section{Summary and Conclusion}
\label{sec:Design:SummaryAndConclusion}
This chapter defined reference Sols and their power budgets from which \ac{SA} design requirements and constraints were fine tuned. Two \ac{SA} designs were proposed, one for each mission site. Their conceptualization was driven by maximizing traverse gains obtained from leveraging the rover's suspension system as a \ac{SA} orientation and inclination mechanism. The results obtained from simple simulations were described which showcased the advantages in solar power output that the proposed inclined driven design offers when compared to a traditional horizontal surface configuration. The simulated data was produced by a custom built \ac{PMS} implemented for as part of this thesis.
