It is not fully understood why differences exist between published and calculated global insolation values considering that, to the best of the author's knowledge, the same input parameters were used in both cases. Not presented in this chapter are differences for the beam, diffuse, and albedo components that compose the global insolation. These do not suggest that a single component is the source of the issue. Analyzing the differences as a function of tau hints that the issue is closely related to tau value inputs. Possible explanations could be:
\begin{itemize}
    \item The normalized net flux function lookup tables was used for insolations presented in \citemarsenv{Appelbaum1990} and \citemarsenv{Appelbaum1993} rather than approxmiated via the polynomial expression.
    \item The albedo values were not approximated in the same manner for insolations presented in \citemarsenv{Appelbaum1993}.
    \item The analytical precision differs between the computational medium used to obtain the published insolations versus the calculations obtained with R.
\end{itemize}

Regardless of the observed differences, the outputs given by the R package will still be used for the analysis in this thesis based on the following justifications:
\begin{itemize}
  \item The calculated global insolation values are lesser than those published. This introduces a conservative element from which mission analysis will benefit in terms of mitigating against the risk of over-predicting power budgets and energy predictions.
  \item The largest differences are for smaller values of tau. Mission analysis will focus on larger tau values as they present the worst-case energy generation outcome with respect to the Martian dust environment.
  \item The differences between calculated and published values are small in terms of the resulting insolation relative to the total daily insolation.
  \item The calculated and published values closely follow the same daily variation pattern.
\end{itemize}
