The thesis is structured in five chapters. The foundation of the thesis are presented in the two chapters following the introduction. Power subsystems of past, ongoing, and planned missions are briefly introduced in \refChpt{sec:StateOfTheArt}. The \ac{SOA} of solar and battery cell technologies built on top of mission heritage are then summarized. \refChpt{sec:MarsSolarEnergy} rigorously explores the Martian solar radiation environment from which two mission sites are selected. Worst case power budget scenarios are presented which impose environmental design constraints.

The proposed \ac{PV} power system is described in \refChpt{sec:PowerSystemDesign}. Reference Sols and their power budgets are formalized in order to fine tune feasable requirements and design drivers. Two \ac{SA} designs are presented for each mission site and a brief simulation prototype is introduced. The thesis is concluded in \refChpt{sec:ConclusionAndOutlook} which provides an outlook for future work.

Additional material is provided in Appendices. Verification of the mars solar insolation R package developed for this thesis is presented in \refApp{sec:Appendix:InsolationCalculationVerification}. The energy predictions calculated by the R package are also validated against \ac{MER} measurements. An attempt to reduce the observed divergences is elaborated in \refApp{sec:Appendix:EnergyErrorMargin}. Finally, \ac{SA} inclination and orientation angles are defined in \refApp{sec:Appendix:OptimalAngles}.
