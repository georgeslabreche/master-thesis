The exploration rover SherpaTT is shown in \refFig{fig:sherpatt}. It has a total mass of approximately \SI{160}{\kilo\gram} and the legs as well as the \ac{RA} weigh about \SI{25}{\kilo\gram} each. The wheeled-leg active suspension system allows the rover to assume different poses with varying inertial moments. The rover is one of many systems comprising a \ac{MRS} developed at the \ac{DFKI}'s \ac{RIC}. The rover has undergone several field trial campaigns, particularly with a Mars analogue terrain field deployment in Utah, USA, where a logistics chain for sample return was evaluated. The rover's versatility has been demonstrated through a multitude of tasks such as assembling surface deployable payloads, deploying BaseCamps, or using its \ac{RA} for soil sampling with modular \ac{PLI} sampling devices \citeother{Cordes2018}.

\vspace{0.4cm}

\begin{figure}[h]
\captionsetup[subfigure]{justification=centering}
%\vspace{-2ex}
\centering
    %% setup sizes
    \setlength{\subfigureWidth}{0.50\textwidth}
    \setlength{\graphicsHeight}{55mm}
    %% kill hyper-link highlighting
    \hypersetup{hidelinks=true}%
    %% the figures
    \begin{subfigure}[t]{\subfigureWidth}
        \centering
            \includegraphics[height=\graphicsHeight]{sections/introduction/background/images/sherpa-tt.png}
            \subcaption{Exploration rover SherpaTT}
            \label{fig:sherpatt}
    \end{subfigure}\hfill
    \begin{subfigure}[t]{\subfigureWidth}
        \centering
            \includegraphics[height=\graphicsHeight]{sections/introduction/background/images/sherpatt-actively-articulated-suspension-sytem.png}
            \subcaption{Actively articulated suspension system}
            \label{fig:sherpatt-actively-articulated-suspension-system}
    \end{subfigure}\\[0.8ex]
    \caption[SherpaTT]
    {SherpaTT and its actively articulated suspension system. Taken from  \citeother{Cordes2018}.}
    \label{}
\vspace{-2ex}
\end{figure}

SherpaTT's actively articulated suspension system consists of four wheeled-legs with a total of 20 motors. The distribution of motors across a single leg is shown in \refFig{fig:sherpatt-actively-articulated-suspension-system}. Each leg is equipped with three suspension motors and two drive motors. The suspension motors are responsible for Pan, \ac{IL}, and \ac{OL} revolute joint rotations whereas the drive motors are responsible for \ac{WS} and \ac{WD}.
