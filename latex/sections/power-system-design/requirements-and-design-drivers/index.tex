Requirements and design drivers constrain the rover's \ac{SA} design. This section reviews and complements the assumptions made thus far. The selected solar cell technology and the assumptions pertaining to its \ac{BOL} and \ac{EOL} efficiencies are also presented. Further assumptions are made regarding the rover's mobility capabilities.

\subsection{Assumptions}
\label{sec:RequirementsAndDesignDrivers:Assumptions}
The assumptions are:

\resetLeadingZeroCounter
\begin{enumerate}[leftmargin=1.31cm, label=\zeroLeadCounter{A}]
    \item Propulsion power draw for flat terrain traverses is \SI{75}{\watt}.
    \item Propulsion power draw for upslope terrain traverses up to \SI{30}{\degree} inclination is \SI{150}{\watt}.
    \item Solar cell technology is \ac{IMM} solar cell with AM0 efficiency of \SI{32.3}{\percent} at \ac{BOL} and \SI{22.5}{\percent} at \ac{EOL} \citepower{Sharps2017}.
    \item \label{itm:ass:solar_cell_efficiency} \ac{IMM} Solar cell efficiency on Mars surface is \SI{22}{\percent} at \ac{EOL}.
    \item \label{itm:ass:packing_efficiency} Solar cell packing efficiency is \SI{85}{\percent}.
    \item \label{itm:ass:red_shifts} Solar cell efficiency varies by \SI{3}{\percent} due to changing temperature and red-shift spectral losses through the day time-period \citepower{Kerslake1999}.
    \item \label{itm:ass:dust_deposition_saturation} \ac{SA} dust performance degradation saturates at \SI{30}{\percent}\citepower{Stella2005}.
    \item \label{itm:ass:protruding_shadowing} \ac{SA} performance degradation from shadowing by protruding rover structures is \SI{5}{\percent}.
    \item \label{itm:ass:sa_surface_density} \ac{SA} surface density is \SI{3.7}{kg.m^{-2}}.
    \item The rover's nominal speed is at least \SI{2}{cm.s^{-1}}.
    \item There is an average \SI{15}{\percent} slippage when traversing a flat terrains for any given distance \citeother{Cordes2018}.
    \item There is an average \SI{30}{\percent} slippage when traversing an upslope terrain for any given distance and slope angle \citeother{Cordes2018}.
\end{enumerate}

\subsection{Requirements}
\label{sec:sec:Design:RequirementsAndDesignDrivers:Requirements}
The following considerations are taken into account to narrow down the requirements:

\begin{enumerate}[label=\textcolor{BulletBlue}{(\alph*)}]
    \item Interest in long traverses from the PERASPERA programme on space robotics \citeother{PERASPERA}.
    \item Focus on space robotics development for long autonomous traverses at \ac{DFKI} \citeother{OG6} \citeother{OG10}.
    \item Mission site at Iani Chaos is approximately \SI{100}{\kilo\meter} from closest resource deposits of interest.
    \item Area of interest at Ismenius Cavus is approximately $\SI{100}{\kilo\meter} \times \SI{50}{\kilo\meter}$.
    \item Dust storm tracking in \citemarsenv{Battalio2019} observed that the average duration of local dust storms is approximately seven Sols.
\end{enumerate}

The following \ac{EOL} requirements apply for a mission life time of one \ac{MY}:

\resetLeadingZeroCounter
\begin{enumerate}[leftmargin=1.30cm, label=\zeroLeadCounter{R}]
    %\item The rover shall be able to traverse flat terrain at an optical depth of $\tau = 1$.
    %\item The rover shall be able to traverse \SI{30}{\degree} inclined terrain at an optical depth of $\tau = 1$.
    \item \label{itm:req:total_distance_flat_terrain} The rover shall traverse flat terrain for a total distance of at least \SI{10}{\kilo\meter}.
    \item \label{itm:req:survive_tau1} The rover shall survive optical depths of up to $\tau = 1$.
    \item \label{itm:req:survice_tau2} The rover shall survive optical depths $\tau = 2$ for at least seven Sols.
\end{enumerate}

\subsection{Constraint}
\label{sec:Design:RequirementsAndDesignDrivers:Constraints}
The design must allow for \textit{non-Traverse Sols} such as long science stops, intermittent hibernation periods during dust storms, battery recharging periods, or limited activities during a solar conjunction break. Furthermore, traverses may only occur during daylight to allow terrain observations to be made. The following constraints are considered:

\resetLeadingZeroCounter
\begin{enumerate}[leftmargin=1.47cm, label=\zeroLeadCounter{C}]
    \item There shall be a minimum of at least two \textit{non-Traverse Sols} between each \textit{Traverse Sol}.
    \item \label{itm:con:daylight_traverse} The rover can be in \textit{Traverse mode} only during daylight.
\end{enumerate}

\subsection{Design Drivers}
\label{sec:Design:RequirementsAndDesignDrivers:DesignDrivers}
The \ac{SA} design drivers are:

\resetLeadingZeroCounter
\begin{enumerate}[leftmargin=1.51cm, label=\zeroLeadCounter{D}]
    \item \label{itm:dd:limited_unfolding} Limited amount of unfolding during deployment.
    \item \label{itm:dd:shadowing} Shadowing is minimized.
    \item \label{itm:dd:end_effector} Manipulator arm end effector can access the ground.
    \item \label{itm:dd:four_plis} Manipulator arm can access at least four \ac{PLI} stored in the rover.
    \item \label{itm:dd:unobstructed} Movement range of the suspension system is unobstructed.
    \item \label{itm:dd:cog} Position of the rover's \ac{CoG} along the x and y axis is centered during stowed position.
\end{enumerate}

%\subsection{Summary}
%\label{sec:Design:RequirementsAndDesignDrivers:Summary}
%The assumptions, requirements, constraints, and design drivers presented in this section apply to both missions sites and are used to size the rover's \acp{SA}.
