The findings of this thesis support using the SherpaTT rover active suspension system as a mechanism for solar panel inclination and orientation under certain \ac{SA} sizing conditions. Combining a maximum attainable \ac{SA} surface inclination angle of \SI{10}{\degree} with optimal daily orientation angles results in appreciable traverse gains when compared to what is obtained from a horizontal surface. This is not true for large \ac{SA} areas and the smaller the required solar cell coverage area the greater the gains. In light of this, \ac{SA} sizing for the the worst case daily insolations introduce a hibernation mode solar power draw constraint of \SI{17}{\watt} near the equator at \SI{2}{\degree}S and \SI{15}{\watt} in the northern hemisphere at \SI{34}{\degree}N. With these power budgets, an average traverse gain of \SI{34}{\percent} is achieved at \SI{2}{\degree}S for a clear day with $\tau$ factor 0.4. At \SI{34}{\degree}N, the average gain is \SI{25}{\percent} under the same atmospheric opacity. These gains remain significant on dusty days at \SI{23}{\percent} and \SI{10}{\percent} at respective latitudes with $\tau$ factor 1.

A redesigned body is proposed for SherpaTT to accomodate the required \acp{SA} for the selected mission sites. Solar panel deployment sequences and motor sizing are also considered. The placement of the rover's \ac{RA} prevents the solar panels from resting on top of the rover's body during their stowed configuration. This is problematic with respect to the limited payload space available in a \ac{LV} as well as with design constraints that are imposed by a lander. Tackling this issue requires the \ac{RA}'s location and stowed configuration to be re-evaluated, in particular with respect to potential obstructions with the stowed position of the rover's legs. Further work is required so that future versions of the rover may be designed to navigate extra-terrestial planetary surfaces. Alternative use of the wheeled leg system as a means of obtaining traverse gains is demonstrated through simulation. However, a more robust analysis is required to determine if these gains remain relevant when compared to the suspension system power draws required to maintain the power optimal \ac{SA} inclination and orientation configurations.

This thesis presented a suspension system use case that goes beyond the scenario of negotiating complex terrains such as steep slopes or exploring crater environments. Expanding on this idea widens the range of potential research work, such as in:
\begin{itemize}
  \item Achieving higher pitch and rolls angles so that higher \ac{SA} inclination angles may be attained.
  \item Simulating ground adaption scenarios that preserve desired \ac{SA} configurations into a fixed plane.
  \item Simulating shadowing events on the \ac{SA} and analyzing how they affect power outputs, in particular for shadows caused by the \ac{RA}.
  \item Extending the \ac{PMS} into a plugin for the MARS simulation platform or as a component of the SherpaTT Control GUI.
  \item Developing a \ac{PV} plane object ``phobostype'' as well as the associated Rock tasks so that solar panels may be modeled in Blender/Phobos and exported as sensors into MARS.   \item Including \ac{IMU} data in the \ac{PV} plane object ``phobostype'' so that panels may be used as independent power units with their own inclinations and orientations configurations.
  \item Implementing a battery model to simulate battery charge and discharge for a more complete power subsystem simulation.
  \item Developing power budgets for onboard instruments to simulate scenarios such as drilling for sample return missions.
  \item Using concurrent engineering to explore other aspects of Mars mission planning such as thermal management, \ac{LV} selection, trajectory analysis, telecommunication, and \ac{EDL}.
  \item Applying lessons learned from this thesis to propose SherpaTT \ac{SA} designs for terrestrial and lunar applications.
  \item Prototyping \ac{SA} on SherpaTT in a lab setting and in field trials.
  \item Exploring other alternative use cases for the rover's suspension system such as star tracking for astronomical observations with a mounted telescope.
\end{itemize}
