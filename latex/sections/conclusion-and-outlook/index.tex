The findings presented in this thesis are fundamentally aimed at reaching higher \acp{TRL} for SherpaTT. A high level design of a solar power subsystem is proposed that enables long traverses at two different Martian sites. Further work is required so that future versions of the rover may be designed to navigate the topography of extra-terrestial planetary surfaces. Alternative use of the wheeled leg system as a means of obtaining solar power output gains have been projected and demonstrated through simulation. However, a more robust analysis is required to determine if these gains are relevant when compared to the suspension system power draws incurred by maintaining the power optimal \ac{SA} inclination and orientation configuration during an entire Sol.

This thesis presented a suspension system use case that goes beyond the obvious scenario of negotiating complex terrain such as steep slopes or exploraing crate environments. Expanding on this idea widens the range of potential research work such as:

\begin{itemize}
  \item Simulating ground adaption scenarios that preserve desired \ac{SA} configurations into a fixed plane.
  \item Formalizing and implementing  inclination angle $\beta$ and orientation angle $\gamma_{c}$ determinations when the rover executes dual-axis pitch and roll repositioning manoeuvers.
  \item Extending the \ac{PMS} into a plugin for the MARS simulation platform or as a component of the SherpaTT Control GUI.
  \item Developing a \ac{PV} plane object ``phobostype'' as well as the associated rock tasks so that solar panels may be modeled and exported as sensors in Blender/Phobos. Embed \ac{IMU} sensors so that panels may be used as independent power units with their own inclinations and orientations configurations.
  \item Implementing a battery model to simulate battery charge and discharge as part of a more complete power subsystem.
  \item Developing power budgets for onboard instruments to simulate mission scenarios such as drilling for sample collection.
  \item Using concurrent engineering to explore other aspects of Mars mission planning such as thermal management, \ac{LV} selection, trajectory analysis, telecommunication, and \ac{EDL}.
  \item Applying lessons learned from this thesis to propose SherpaTT \ac{SA} designs for terrestial and lunar applications.
  \item Prototyping solar arrays on SherpaTT in a lab setting and in field trials.
\end{itemize}
