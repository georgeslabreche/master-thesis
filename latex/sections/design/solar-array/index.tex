\ac{SA} sizing is driven by the worst case daily insolation under which the rover must still be able to operate. The process is approached with an initial worst case definition which is then refined to satisfy traverse requirements all while reducing the \ac{PV} system's size and mass.

\subsection{Sizing}
\refEqn{eq:SA_slope_energy} is rearranged into an expression of the solar cell coverage area $A$, shown in \refEqn{eq:solar_cell_coverage_area}:

\begin{equation}
  \label{eq:solar_cell_coverage_area}
  A = \frac{E}{\eta \cdot H \cdot PR}
\end{equation}

where $E$ is the energy required by the rover and $H$ is the worst-case available daily insolation. The worst case energy requirement is denoted as $E_{req}^{worst}$. An inclined surface is used for the worst case daily insolation rather than a horizontal surface hence $H_{\beta}^{worst}$ is considered instead of $H_{h}^{worst}$. The following values are used:

\begin{enumerate}[label=\textcolor{BulletBlue}{(\alph*)}]
    \item From \refTab{tab:worst-case-traverse-sol-power-budget}, $E_{req}^{worst}$ is \SI{775}{\watt\hour} at Iani Chaos and \SI{750}{\watt\hour} at Ismenius Cavus.
    \item From \refTab{tab:insolation-iani-chaos-clear-and-dusty-days} and \refTab{tab:insolation-ismenius-cavus-clear-and-dusty-days}, $H_{\beta}^{worst}$ for $\tau=1$ is \SI{2479}{Wh.m^{-2}} at Iani Chaos and \SI{1345}{Wh.m^{-2}} at Ismenius Cavus.
    \item From \ref{itm:ass:red_shifts}, \ref{itm:ass:dust_deposition_saturation}, and \ref{itm:ass:protruding_shadowing}, \ac{PR} at \ac{EOL} is $PR_{EOL} = 1 - (0.03 + 0.3 + 0.05) = 0.62$.
    \item From \ref{itm:ass:solar_cell_efficiency}, $\eta_{EOL} = 0.22$.
\end{enumerate}


The required solar cell coverage area at Iani Chaos is:
\begin{align}
  \label{calc:solar_cell_area_iani_chaos_traverse}
  A_{iani} &= \frac{E_{req}^{worst}}{\eta_{EOL} \cdot H_{\beta}^{worst} \cdot PR_{EOL}}\\
           &= \frac{775}{0.22 \cdot 2479 \cdot 0.62}\\
           &= \SI{2.29}{m^{2}}
\end{align}

and at Ismenius Cavus:
\begin{align}
  \label{calc:solar_cell_area_ismenius_cavus_traverse}
  A_{ismenius} &= \frac{E_{req}^{worst}}{\eta_{EOL} \cdot H_{\beta}^{worst} \cdot PR_{EOL}}\\
               &= \frac{750}{0.22 \cdot 1345 \cdot 0.62}\\
               &= \SI{4.09}{m^{2}}
\end{align}

Considering the solar cell packing efficiency \ref{itm:ass:packing_efficiency} results in a \ac{SA} area of \SI{2.7}{m^{2}} at Iani Chaos. from \ref{itm:ass:sa_surface_density}, the \ac{SA} mass is 9.95 \si{\kilo\gram}. \ac{SA} inclination with $\beta_{best} = \SI{10}{\degree}$ results in a \ac{SA} sizing decrease of \SI{3.9}{\percent} with respect to the horizontal surface alternative. The total maximum flat traverse distance achievable over the course of one \ac{MY} is increased by \SI{13.22}{\percent} from \SI{42.43}{\kilo\meter} to \SI{48.04}{\kilo\meter} when $\tau = 1$ during global dust storm season and $\tau = 0.4$ for the remainder of the year. At Ismenius Cavus, the \ac{SA} area is \SI{4.8}{m^{2}} and its mass 17.8 \si{\kilo\gram}. A \SI{4.6}{\percent} \ac{SA} size decrease is achieved when inclination is taken into account. The total maximum achievable flat traverse distance during one \ac{MY} is increased by \SI{2.13}{\percent} from \SI{63.05}{\kilo\meter} to \SI{64.39}{\kilo\meter}. The traverse distance gains attributed to \ac{SA} inclination capabilities do not justify adopting the complexities of an active suspension system for the purpose of increasing traverse distance. This is particularly true at Ismenius Cavus. The savings in \ac{SA} size and mass also leave much to be desired. Furthermore, the large \acp{SA} are problematic for rover integration. This is illustrated in \refFig{fig:sa-area-initial-sizes}.

\begin{figure}[h]
  \captionsetup[subfigure]{justification=centering}
  \centering
  \hypersetup{linkcolor=captionTextColor}
  \includegraphics[width=0.7\linewidth]{sections/design/solar-array/images/sa-area-initial-sizes.png}\\
  \caption[Initial solar array sizing for mission sites]
          {Initial \ac{SA} sizing for mission sites. The outlined square areas are equivalent to \ac{SA} areas of \SI{2.7}{m^{2}} for Iani Chaos and \SI{4.8}{m^{2}} for Ismenius Cavus.}
  \label{fig:sa-area-initial-sizes}
\end{figure}

To explain the lack of significant gains with $\beta_{best}$, the generated \ac{SA} energy and maximum traverse durations are plotted in \refFig{fig:plot:iani-chaos-generated-energy-and-max-traverse-durations} and \refFig{fig:plot:ismenius-cavus-generated-energy-and-max-traverse-durations}. At Ismenius Cavus, the \ref{itm:con:daylight_traverse} constraint imposes an upper limit to the maximum traverse durations. The ceiling corresponded to the daylight time that is available for traversing. This results in the \ac{SA} generating excess propulsion energy which cannot be used. As seen in \refFig{fig:plot:sub:ismenius-cavus-max-traverse-durations}, an inclined \ac{SA} is only relevant during the global dust storm season with a dusty atmosphere of $\tau = 1$. This negligeable advantage results in the limited total traverse gains obtained with $\beta_{best}$. Lack of appreciable traverse distance gains obtained with $\beta_{best}$ is due to the $\tau$ factor applied when evaluating the $H_{\beta}^{worst}$ daily insolation with $\tau = 1$. At high optical opacities, $H_{\beta}^{worst} \approx H_{h}^{worst}$ due to light scattering by airborn Martian dust. This was previously observed in \refFig{fig:plot:insolation-ls}, where diffuse insolation was already the largest component contributing to global insolation at $\tau = 1$. Inclined \ac{SA} surface yield little to no benefits over horizontal surfaces when light is mostly diffuse.

\begin{figure}[h]
\captionsetup[subfigure]{justification=centering}
\vspace{-2ex}
	\centering
    %% setup sizes
    \setlength{\subfigureWidth}{0.50\textwidth}
    \setlength{\graphicsHeight}{80mm}
    %% kill hyper-link highlighting
    \hypersetup{hidelinks=true}%
    %% the figures
    \begin{subfigure}[t]{\subfigureWidth}
        \centering
        \includegraphics[height=\graphicsHeight]{sections/design/solar-array/plots/ianichaos-daily-generated-energy-for-solar-cell-coverage-area-23m2.png}
        \subcaption{Generated Energy}
        \label{fig:plot:sub:iani-chaos-generated-energy}
    \end{subfigure}\hfill
    \begin{subfigure}[t]{\subfigureWidth}
        \centering
        \includegraphics[height=\graphicsHeight]{sections/design/solar-array/plots/ianichaos-75w-max-traverse-durations-for-solar-cell-coverage-area-23m2.png}
  		\subcaption{Maximum Traverse Durations}
		\label{fig:plot:sub:iani-chaos-max-traverse-durations}
	\end{subfigure}\\[0.8ex]
    \caption[Generated energy and maxium flat terrain traverse durations at Iani Chaos]
            {Generated energy and maxium flat terrain traverse duration at Iani Chaos with solar cell coverage area of \SI{2.3}{m^{2}}. Optical depth  $\tau = 1$ was used for global dust storm season ($\SI{185}{\degree} \leq L_{s} \leq \SI{315}{\degree}$) and $\tau = 0.4$ for the remainder of the year. The \textit{available daylight traverse time} corresponds to the amount of daylight hours left in a \textit{Traverse Sol} after subtracting the time taken by non-Traverse modes: \textit{Idle - Day}, \textit{\ac{DTE} Communication}, \textit{Science Stop - Short}, and \textit{Optimal Pose}. The maximum traverse durations for \ac{SA} horizontal do not consider the \textit{Optimal Pose} mode.}
    \label{fig:plot:iani-chaos-generated-energy-and-max-traverse-durations}
\vspace{-2ex}
\end{figure}

\begin{figure}[h]
\captionsetup[subfigure]{justification=centering}
\vspace{-2ex}
	\centering
    %% setup sizes
    \setlength{\subfigureWidth}{0.50\textwidth}
    \setlength{\graphicsHeight}{80mm}
    %% kill hyper-link highlighting
    \hypersetup{hidelinks=true}%
    %% the figures
    \begin{subfigure}[t]{\subfigureWidth}
        \centering
        \includegraphics[height=\graphicsHeight]{sections/design/solar-array/plots/ismeniuscavus-daily-generated-energy-for-solar-cell-coverage-area-41m2.png}
        \subcaption{Generated Energy}
        \label{fig:plot:sub:ismenius-cavus-generated-energy}
    \end{subfigure}\hfill
    \begin{subfigure}[t]{\subfigureWidth}
        \centering
        \includegraphics[height=\graphicsHeight]{sections/design/solar-array/plots/ismeniuscavus-75w-max-traverse-durations-for-solar-cell-coverage-area-41m2.png}
  		\subcaption{Maximum Traverse Durations}
		\label{fig:plot:sub:ismenius-cavus-max-traverse-durations}
	\end{subfigure}\\[0.8ex]
    \caption[Generated energy and maxium flat terrain traverse durations at Ismenius Cavus]
            {Generated energy and maxium flat terrain traverse duration at Ismenius Cavus with solar cell coverage area of \SI{4.1}{m^{2}}. The same considerations were taken as in Figure \ref{fig:plot:iani-chaos-generated-energy-and-max-traverse-durations}.}
    \label{fig:plot:ismenius-cavus-generated-energy-and-max-traverse-durations}
\vspace{-2ex}
\end{figure}

\clearpage
\ac{SA} sizes of \SI{2.7}{m^{2}} at Iani Chaos and \SI{4.8}{m^{2}} at Ismenius Cavus results in total attainable flat traverse coverages of \SI{48.04}{\kilo\meter} and \SI{64.39}{\kilo\meter}, respectively, for one \ac{MY}. These performances exceed by approximately five- and six-folds the \ref{itm:req:total_distance_flat_terrain} requirement of \SI{10}{\kilo\meter}. With this in mind, \ac{SA} sizing using \refEqn{calc:solar_cell_area_iani_chaos_traverse} and \refEqn{calc:solar_cell_area_ismenius_cavus_traverse} are approached differently. $E_{req}^{worst}$ is changed from energy required for the worst case \textit{Traverse Sol} to energy required for a \textit{Hibernation Sol}. However, to obtain satisfying results the solar power draw required by the \textit{Hibernation mode} is reduced from \SI{18}{\watt} to \SI{17}{\watt} at Iani Chaos and from \SI{18}{\watt} to \SI{15}{\watt} at Ismenius Cavus. Introducing \acp{RHU} is necessary should these revised solar power requirements be unachievable. A single \ac{RHU} provides approximately \SI{1}{\watt} of heat thus one \ac{RHU} is required at Iani Chaos and three at Ismenius Cavus. Taking into account a \SI{20}{percent} system margin, $E_{req}^{worst}$ becomes \SI{490}{\watt\hour} for a \textit{Hibernation Sol} at Iani Chaos and \SI{432}{\watt\hour} at Ismenius Cavus. Revisting \refEqn{eq:solar_cell_coverage_area} results in the following solar cell coverage areas:


\begin{align}
  \label{calc:solar_cell_area_iani_chaos_hibernation}
  A_{iani} &= \frac{E_{req}^{worst}}{\eta_{EOL} \cdot H_{\beta}^{worst} \cdot PR_{EOL}}\\
           &= \frac{490}{0.22 \cdot 2479 \cdot 0.62}\\
           &= \SI{1.45}{m^{2}}
\end{align}

\begin{align}
  \label{calc:solar_cell_area_ismenius_cavus_hibernation}
  A_{ismenius} &= \frac{E_{req}^{worst}}{\eta_{EOL} \cdot H_{\beta}^{worst} \cdot PR_{EOL}}\\
               &= \frac{432}{0.22 \cdot 1345 \cdot 0.62}\\
               &= \SI{2.35}{m^{2}}
\end{align}

After considering \ref{itm:ass:packing_efficiency}, the \ac{SA} areas are \SI{1.7}{m^{2}} at Iani Chaos with a mass of \SI{6.3}{\kilo\gram} and \SI{2.8}{m^{2}} at Ismenius Cavus with a mass of  \SI{10.4}{\kilo\gram}. The \ref{itm:req:total_distance_flat_terrain} requirement remain satisfied with
a maxmium flat traverse distance coverage of \SI{12}{\kilo\meter} at Iani Chaos and \SI{35}{\kilo\meter} at Ismenius Cavus. \refFig{fig:plot:flat-traverse-gains-for-different-sa-area} compares the gains in flat traverse distance coverage for different solar cell coverage areas and optical depths in order to better appreciate the gains involved. At Iani Chaos, a \SI{34}{\percent} gain in traverse is achieved on a clear day with $\tau = 0.4$. The gain remains significant at \SI{23}{\percent} for a dusty day with $\tau = 1$. At Ismenius cavus, $\tau = 0.4$ and $\tau = 1$ gains are \SI{19}{\percent} and \SI{11}{\percent}, respectively. The maximum daily traverse durations attainable at both sites with their respective solar cell coverage areas are shown in \refFig{fig:plot:final-maximum-traverse-durations-at-missions-sites}.

\begin{figure}[h]
\captionsetup[subfigure]{justification=centering}
\vspace{-2ex}
	\centering
    %% setup sizes
    \setlength{\subfigureWidth}{0.50\textwidth}
    \setlength{\graphicsHeight}{80mm}
    %% kill hyper-link highlighting
    \hypersetup{hidelinks=true}%
    %% the figures
    \begin{subfigure}[t]{\subfigureWidth}
        \centering
        \includegraphics[height=\graphicsHeight]{sections/design/solar-array/plots/ianichaos-75w-traverse-gains-for-different-solar-cell-coverage-areas.png}
  		\subcaption{Iani Chaos}
		\label{fig:plot:sub:ismenius-chaos-flat-traverse-gains-for-different-sa-area}
    \end{subfigure}\hfill
    \begin{subfigure}[t]{\subfigureWidth}
        \centering
        \includegraphics[height=\graphicsHeight]{sections/design/solar-array/plots/ismeniuscavus-75w-traverse-gains-for-different-solar-cell-coverage-areas.png}
  		\subcaption{Ismenius Cavus}
		\label{fig:plot:sub:iani-chaos-flat-traverse-gains-for-different-sa-area}
	\end{subfigure}\\[0.8ex]
    \caption[Flat traverse distance gains at mission sites for different solar cell coverage areas]
            {Flat traverse distance gains at mission sites for different solar cell coverage areas.}
    \label{fig:plot:flat-traverse-gains-for-different-sa-area}
\vspace{-2ex}
\end{figure}


\begin{figure}[h]
\captionsetup[subfigure]{justification=centering}
\vspace{-2ex}
	\centering
    %% setup sizes
    \setlength{\subfigureWidth}{0.50\textwidth}
    \setlength{\graphicsHeight}{80mm}
    %% kill hyper-link highlighting
    \hypersetup{hidelinks=true}%
    %% the figures
    \begin{subfigure}[t]{\subfigureWidth}
        \centering
        \includegraphics[height=\graphicsHeight]{sections/design/solar-array/plots/ianichaos-75w-max-traverse-durations-for-solar-cell-coverage-area-15m2.png}
  		\subcaption{Iani Chaos, solar cell coverage = \SI{1.5}{m^{2}}}
		\label{fig:plot:sub:final-maximum-traverse-durations-iani-chaos}
    \end{subfigure}\hfill
    \begin{subfigure}[t]{\subfigureWidth}
        \centering
        \includegraphics[height=\graphicsHeight]{sections/design/solar-array/plots/ismeniuscavus-75w-max-traverse-durations-for-solar-cell-coverage-area-24m2.png}
  		\subcaption{Ismenius Cavus, solar cell coverage area = \SI{2.4}{m^{2}}}
		\label{fig:plot:sub:final-maximum-traverse-durations-ismenius-cavus}
	\end{subfigure}\\[0.8ex]
    \caption[Maximum traverse durations at mission sites]
            {Maximum traverse durations at mission sites.}
    \label{fig:plot:final-maximum-traverse-durations-at-missions-sites}
\vspace{-2ex}
\end{figure}


%\clearpage
%\subsubsection{Battery}
%\todo[inline]{\textbf{TODO:} Battery size based on energy required to keep the rover Warm through the night. Check if the calculated size satisfies the hibernation requirement. If not, resize.}

\clearpage
\subsection{Baseline Design}
As per \ref{itm:dd:shadowing}, the rover's body is redesigned in order to minimize shadowing events. A pyramid shaped body continuously casts a shadow on a surrounding \ac{SA} with an exception during solar noon. Even then, it is likely that the rover's tilt still subjects its \ac{SA} panels to shadowing. The redesigned body shown in \refFig{fig:rover-body-redesign} opts for a box shape with a flat top of which the forward \ac{SA} panel rests. The dimension of the redesigned body is $\SI{65}{\centi\meter}\times\SI{65}{\centi\meter}\times\SI{63}{\centi\meter}$. The base dimension is the same as that of the pyramid's in order to minimize changes in sizing dependencies with the overall system.

\begin{figure}[h]
\captionsetup[subfigure]{justification=centering}
\vspace{-2ex}
	\centering
    %% setup sizes
    \setlength{\subfigureWidth}{0.50\textwidth}
    \setlength{\graphicsHeight}{48mm}
    %% kill hyper-link highlighting
    \hypersetup{hidelinks=true}%
    %% the figures
    \begin{subfigure}[t]{\subfigureWidth}
        \centering
        \includegraphics[height=\graphicsHeight]{sections/design/solar-array/images/body-before.png}
        \subcaption{Before}
		\label{fig:sub:rover-body-redesign-before}
    \end{subfigure}\hfill
    \begin{subfigure}[t]{\subfigureWidth}
        \centering
        \includegraphics[height=\graphicsHeight]{sections/design/solar-array/images/body-after.png}
  		\subcaption{After}
		\label{fig:sub:rover-body-redesign-after}
	\end{subfigure}\\[0.8ex]
    \caption[Rover body redesign]
            {Rover body redesign.}
    \label{fig:rover-body-redesign}
\vspace{-2ex}
\end{figure}

Placing the four \acp{PLI} is driven by \ref{itm:dd:four_plis} and shown in \refFig{fig:sub:rover-body-redesign-after}. They are stored in the front of body and are accessible by the rover's \ac{RA}. As is seen further in this section, access to the body's other faces is obstructed by the \ac{SA} panels. Spacing is left between the \acp{PLI} and the top of the body to allow for instruments. The \acp{PLI} are oriented upwards by \SI{15}{\degree} in order to compensate against forward body pitches. The upward tilt of the \ac{PLI} has the added benefit of facilitated access for the \ac{RA}. The position of the latter is shifted towards the front in order to maximize the surface area that can be uninterruptedly covered by solar cells. As per \ref{itm:dd:unobstructed}, the movemement range of the suspension system must remain unobstructed. The height of the redesigned body is such that the deployed \ac{SA} panels are kept above the legs for any configuration of the active suspension system. This is demonstrated in \refFig{fig:rover-body-redesign-stowed-legs} where the rover's knees reach their highest point when stowed.

\begin{figure}[h]
\captionsetup[subfigure]{justification=centering}
\vspace{-2ex}
	\centering
    %% setup sizes
    \setlength{\subfigureWidth}{0.50\textwidth}
    \setlength{\graphicsHeight}{50mm}
    %% kill hyper-link highlighting
    \hypersetup{hidelinks=true}%
    %% the figures
    \begin{subfigure}[t]{\subfigureWidth}
        \centering
        \includegraphics[height=\graphicsHeight]{sections/design/solar-array/images/stowed-body-pyramid.png}
        \subcaption{Before}
		\label{fig:sub:rover-body-redesign-stowed-before}
    \end{subfigure}\hfill
    \begin{subfigure}[t]{\subfigureWidth}
        \centering
        \includegraphics[height=\graphicsHeight]{sections/design/solar-array/images/stowed-body-box.png}
  		\subcaption{After}
		\label{fig:sub:rover-body-redesign-stowed-legs-after}
	\end{subfigure}\\[0.8ex]
    \caption[Rover body redesign with stowed legs]
            {Rover body redesign with stowed legs.}
    \label{fig:rover-body-redesign-stowed-legs}
\vspace{-2ex}
\end{figure}


Folded and deployed \ac{SA} panels are show in \refFig{fig:solar-array-on-rover-iani-chaos} and \refFig{fig:solar-array-on-ismenius-cavus-chaos} as they would appear mounted on top of the rover's body for different mission sites. When folded, the port, starboard, and stern panels do not lay on top of the rover's body and are instead kept upright normal to the body's top surface. The base of the \ac{RA} occupies an area of the body's surface that prevents the panels from folding in completely. Shifting the \ac{RA} onto its own support in front of the body would resolve this but prevents the rover's front right leg from assuming its stowed position.

\vspace{0.5cm}

\begin{figure}[h]
\captionsetup[subfigure]{justification=centering}
\vspace{-2ex}
	\centering
    %% setup sizes
    \setlength{\subfigureWidth}{0.50\textwidth}
    \setlength{\graphicsHeight}{53mm}
    %% kill hyper-link highlighting
    \hypersetup{hidelinks=true}%
    %% the figures
    \begin{subfigure}[t]{\subfigureWidth}
        \centering
        \includegraphics[height=\graphicsHeight]{sections/design/solar-array/images/iani-chaos-stowed.png}
  		\subcaption{Folded on stowed rover}
		\label{fig:sub:solar-array-on-rover-for-iani-chaos-stowed}
    \end{subfigure}\hfill
    \begin{subfigure}[t]{\subfigureWidth}
        \centering
        \includegraphics[height=\graphicsHeight]{sections/design/solar-array/images/iani-chaos-10deg-pitch.png}
  		\subcaption{Deployed on rover with \SI{10}{\degree} pitch forward}
		\label{fig:sub:solar-array-on-rover-for-iani-chaos-deployed}
	\end{subfigure}\\[0.8ex]
    \caption[Solar array on rover for Iani Chaos deployment]
            {\ac{SA} on rover for Iani Chaos deployment.}
    \label{fig:solar-array-on-rover-iani-chaos}
\vspace{-2ex}
\end{figure}

\vspace{0.5cm}

The folded configurations shown in \refFig{fig:sub:solar-array-on-rover-for-iani-chaos-stowed} and \refFig{fig:sub:solar-array-on-rover-for-iani-chaos-deployed} were accepted considering that the gain in height does not add much to what is already imposed by the stowed \ac{RA}. In terms of compactness, a preferable configuration would be one in which all panels are folded in and the \ac{RA} is stowed against the front side of the body. However, supporting already existing rover postures is preferred. Future design iterations of the stowed configuration for both the \ac{RA} and the \ac{SA} panels should prioritize compactness based on a \ac{LV} payload capacity and the rover's lander constraints.

\vspace{0.5cm}

\begin{figure}[h]
\captionsetup[subfigure]{justification=centering}
\vspace{-2ex}
	\centering
    %% setup sizes
    \setlength{\subfigureWidth}{0.50\textwidth}
    \setlength{\graphicsHeight}{53mm}
    %% kill hyper-link highlighting
    \hypersetup{hidelinks=true}%
    %% the figures
    \begin{subfigure}[t]{\subfigureWidth}
        \centering
        \includegraphics[height=\graphicsHeight]{sections/design/solar-array/images/ismenius-cavus-stowed.png}
  		\subcaption{Folded on stowed rover}
		\label{fig:sub:solar-array-on-rover-for-ismenius-cavus-stowed}
    \end{subfigure}\hfill
    \begin{subfigure}[t]{\subfigureWidth}
        \centering
        \includegraphics[height=\graphicsHeight]{sections/design/solar-array/images/ismenius-cavus-10deg-pitch.png}
  		\subcaption{Deployed on rover with \SI{10}{\degree} pitch forward}
		\label{fig:sub:solar-array-on-rover-for-ismenius-cavus-deployed}
	\end{subfigure}\\[0.8ex]
    \caption[Solar array on rover for Ismenius Cavus deployment]
            {\ac{SA} on rover for Ismenius Cavus deployment.}
    \label{fig:solar-array-on-ismenius-cavus-chaos}
\vspace{-2ex}
\end{figure}

\clearpage
The unfolded \ac{SA} panel layouts are shown in \refFig{fig:solar-array-layouts-for-missions-sites}. The \ac{SA} at Iani Chaos consists of four panels whereas a six panel design is used at Ismenius Cavus. The forward panels require a cut to accomodate the base of the rover's \ac{RA}.

\vspace{0.5cm}

\begin{figure}[h]
\captionsetup[subfigure]{justification=centering}
\vspace{-2ex}
	\centering
    %% setup sizes
    \setlength{\subfigureWidth}{0.50\textwidth}
    \setlength{\graphicsHeight}{53mm}
    %% kill hyper-link highlighting
    \hypersetup{hidelinks=true}%
    %% the figures
    \begin{subfigure}[t]{\subfigureWidth}
        \centering
        \includegraphics[height=\graphicsHeight]{sections/design/solar-array/images/solar_array_layout_iani_chaos.png}
  		\subcaption{Iani Chaos, \ac{SA} area = \SI{1.7}{m^{2}}}
		\label{fig:sub:solar-array-layouts-for-iani-chaos}
    \end{subfigure}\hfill
    \begin{subfigure}[t]{\subfigureWidth}
        \centering
        \includegraphics[height=\graphicsHeight]{sections/design/solar-array/images/solar_array_layout_ismenius_cavus.png}
  		\subcaption{Ismenius Cavus, \ac{SA} area = \SI{2.8}{m^{2}}}
		\label{fig:sub:solar-array-layouts-for-ismenius-cavus}
	\end{subfigure}\\[0.8ex]
    \caption[Solar array layouts]
            {\ac{SA} layouts. Green and blue panels are located on the rover's port and starboard, respectively. Port bow and stardboard bow panels are indicated by lighter colors than those on the rover's port quarter and starboard quarter. Light red indicates bow panels which rest over the rover's body. Darker red panels are located on the rover's stern. The yellow dots represent the \ac{CoM} for each panel.}
    \label{fig:solar-array-layouts-for-missions-sites}
\vspace{-2ex}
\end{figure}

\vspace{0.5cm}

Round and diagonal edges negatively affect solar cell packing efficiency and are thus avoided. However, a trade-off is made at Ismenius Cavus due to the large coverage area of the panels and in light of \ref{itm:dd:cog}. Diagonal corners are used for the port and starboard panels to reduce the wingspan of the solar array while preserving the \ac{CoG} within the stowed rover's body and to reduce the panel \ac{CoM} misalignments once deployed. Further \ac{CoG} analysis is required for both configurations with respect to the stern and quarter panels when operating the rover's \ac{RA}. This could result in shifting the position of the port and starboard panels. The surface areas and masses of each panel are presented in Table \ref{tab:solar-panel-surface-areas}. Masses are determined from \ref{itm:ass:sa_surface_density} with a \ac{SA} surface density of \SI{3.7}{kg.m^{-2}}.

\vspace{0.5cm}

\begin{table}[h]
\footnotesize
\centering
\caption[Solar panel surface areas and mass for mission sites]
    {Solar panel surface areas and mass for mission sites.}
\label{tab:solar-panel-surface-areas}
\begin{tabular}{l|c|c|c|c|}
\cline{2-5}
\textbf{} & \multicolumn{2}{c|}{\textbf{Iani Chaos}} & \multicolumn{2}{c|}{\textbf{Ismenius Cavus}} \\ \cline{2-5}
 & \textbf{Surface Area {[}m2{]}} & \textbf{Mass {[}kg{]}} & \textbf{Surface Area {[}m2{]}} & \textbf{Mass {[}kg{]}} \\ \hline
\multicolumn{1}{|l|}{{\color[HTML]{A6DBA0} \textbf{Port bow}}} & {\color[HTML]{A6DBA0} \textbf{0.44}} & {\color[HTML]{A6DBA0} \textbf{1.63}} & {\color[HTML]{A6DBA0} \textbf{0.50}} & {\color[HTML]{A6DBA0} \textbf{1.85}} \\ \hline
\multicolumn{1}{|l|}{{\color[HTML]{5AAE61} \textbf{Port quarter}}} & {\color[HTML]{5AAE61} \textbf{-}} & {\color[HTML]{5AAE61} \textbf{-}} & {\color[HTML]{5AAE61} \textbf{0.50}} & {\color[HTML]{5AAE61} \textbf{1.85}} \\ \hline
\multicolumn{1}{|l|}{{\color[HTML]{F4A582} \textbf{Bow}}} & {\color[HTML]{F4A582} \textbf{0.38}} & {\color[HTML]{F4A582} \textbf{1.41}} & {\color[HTML]{F4A582} \textbf{0.38}} & {\color[HTML]{F4A582} \textbf{1.41}} \\ \hline
\multicolumn{1}{|l|}{{\color[HTML]{D6604D} \textbf{Stern}}} & {\color[HTML]{D6604D} \textbf{0.44}} & {\color[HTML]{D6604D} \textbf{1.63}} & {\color[HTML]{D6604D} \textbf{0.42}} & {\color[HTML]{D6604D} \textbf{1.55}} \\ \hline
\multicolumn{1}{|l|}{{\color[HTML]{92C5DE} \textbf{Starboard bow}}} & {\color[HTML]{92C5DE} \textbf{0.44}} & {\color[HTML]{92C5DE} \textbf{1.63}} & {\color[HTML]{92C5DE} \textbf{0.50}} & {\color[HTML]{92C5DE} \textbf{1.85}} \\ \hline
\multicolumn{1}{|l|}{{\color[HTML]{4393C3} \textbf{Starboard quarter}}} & {\color[HTML]{4393C3} \textbf{-}} & {\color[HTML]{4393C3} \textbf{-}} & {\color[HTML]{4393C3} \textbf{0.50}} & {\color[HTML]{4393C3} \textbf{1.85}} \\ \hline
\multicolumn{1}{|r|}{\textbf{Total}} & \textbf{1.7} & \textbf{6.3} & \textbf{2.8} & \textbf{10.4} \\ \hline
\end{tabular}
\end{table}



\clearpage
\subsection{Mechanisms}
Solar panel deployment sequences are presented in this section. The worst case unfolding is identified with respect to panel mass and traveled rotation distance from which initial motor performance requirements are calculated.

\subsubsection{Deployment Sequence}

\ac{SA} panel deployment at Iani Chaos is straightforward and limited to three unfoldings. Port, bow, and starboard panels are deployed simultaneously while the rover is still in its stowed posture. The sequence is illustrated in \refFig{fig:deployment-sequence-iani-chaos}.

\vspace{0.5cm}

\begin{figure}[h]
\captionsetup[subfigure]{justification=centering}
\vspace{-2ex}
	\centering
    %% setup sizes
    \setlength{\subfigureWidth}{0.32\textwidth}
    \setlength{\graphicsHeight}{30mm}
    %% kill hyper-link highlighting
    \hypersetup{hidelinks=true}%
    %% the figures
	\begin{subfigure}[t]{\subfigureWidth}
        \centering
		\includegraphics[height=\graphicsHeight]{sections/design/solar-array/images/deployment/iani-chaos/solar_array_deployment_iani_chaos_000.png}
		\subcaption{Folded}
		\label{fig:sub:deployment-sequence-iani-chaos-stowed}
	\end{subfigure}\hfill
	\begin{subfigure}[t]{\subfigureWidth}
        \centering
		\includegraphics[height=\graphicsHeight]{sections/design/solar-array/images/deployment/iani-chaos/solar_array_deployment_iani_chaos_030.png}
		\subcaption{Mid-deployment}
		\label{fig:sub:deployment-sequence-iani-chaos-mid}
	\end{subfigure}\hfill
    \begin{subfigure}[t]{\subfigureWidth}
        \centering
		\includegraphics[height=\graphicsHeight]{sections/design/solar-array/images/deployment/iani-chaos/solar_array_deployment_iani_chaos_060.png}
		\subcaption{Deployed}
		\label{fig:sub:deployment-sequence-iani-completed}
	\end{subfigure}
	\caption[Solar array deployment sequence at Iani Chaos]
    {\ac{SA} deployment sequence at Iani Chaos. Total \ac{SA} area is \SI{1.7}{\meter\squared}.}
	\label{fig:deployment-sequence-iani-chaos}
\vspace{-2ex}
\end{figure}

\vspace{0.5cm}

\ac{SA} panel deployment at Ismenius Cavus goes through five unfoldings. The sequence is illustrated in \refFig{fig:deployment-sequence-ismenius-cavus}. Port bow, starboard bow, stern panels are simultenously deployed, completing the first three out of the five unfoldings. The remaining two unfoldings deploy the port quarter and starboard quarter panels.

\vspace{0.5cm}

\begin{figure}[h]
\captionsetup[subfigure]{justification=centering}
\vspace{-2ex}
	\centering
    %% setup sizes
    \setlength{\subfigureWidth}{0.32\textwidth}
    \setlength{\graphicsHeight}{30mm}
    %% kill hyper-link highlighting
    \hypersetup{hidelinks=true}%
    %% the figures
	\begin{subfigure}[t]{\subfigureWidth}
        \centering
		\includegraphics[height=\graphicsHeight]{sections/design/solar-array/images/deployment/ismenius-cavus/solar_array_deployment_ismenius_cavus_000.png}
		\subcaption{Folded panels for stowed posture}
		\label{fig:sub:deployment-sequence-ismenius-cavus-stowed}
	\end{subfigure}\hfill
	\begin{subfigure}[t]{\subfigureWidth}
        \centering
		\includegraphics[height=\graphicsHeight]{sections/design/solar-array/images/deployment/ismenius-cavus/solar_array_deployment_ismenius_cavus_030.png}
		\subcaption{Port bow, starboard bow, and stern mid-deployment}
		\label{fig:sub:deployment-sequence-ismenius-cavus-mid-bow-and-stern}
	\end{subfigure}\hfill
    \begin{subfigure}[t]{\subfigureWidth}
        \centering
		\includegraphics[height=\graphicsHeight]{sections/design/solar-array/images/deployment/ismenius-cavus/solar_array_deployment_ismenius_cavus_060.png}
		\subcaption{Port bow, starboard bow, and stern deployed}
		\label{fig:sub:deployment-sequence-ismenius-cavus-full-bow-and-stern}
	\end{subfigure}\\[0.8ex]
%% 2nd row
	\begin{subfigure}[t]{\subfigureWidth}
        \centering
		\includegraphics[height=\graphicsHeight]{sections/design/solar-array/images/deployment/ismenius-cavus/solar_array_deployment_ismenius_cavus_100.png}
		\subcaption{Port quarter and starboard quarter mid-deployment}
		\label{fig:sub:deployment-sequence-ismenius-cavus-mid-quarter}
	\end{subfigure}\hspace*{2.5cm}
    \begin{subfigure}[t]{\subfigureWidth}
        \centering
		\includegraphics[height=\graphicsHeight]{sections/design/solar-array/images/deployment/ismenius-cavus/solar_array_deployment_ismenius_cavus_130.png}
		\subcaption{Completed deployment}
		\label{fig:sub:deployment-sequence-ismenius-cavus-completed}
	\end{subfigure}
    \caption[Solar array deployment sequence at Ismenius Cavus]
    {\ac{SA} deployment sequence at Ismenius Cavus. Total \ac{SA} area is \SI{2.8}{\meter\squared}.}
	\label{fig:deployment-sequence-ismenius-cavus}
\vspace{-2ex}
\end{figure}

\clearpage
\subsubsection{Motor Requirements}
The worst case motor torque for \ac{SA} deployment occurs when unfolding the port quarter and starboard quarter panels at Ismenius Cavus. This sequence is illustrated from \refFig{fig:sub:deployment-sequence-ismenius-cavus-full-bow-and-stern} to \refFig{fig:sub:deployment-sequence-ismenius-cavus-completed}. The distance between the \ac{CoM} and the rotation joint is shown in \refFig{fig:ismenius-cavus-solar-panel-starboard-quarter-dimensions} and corresponds to $r = \SI{385}{\milli\metre}$.

\vspace{0.25cm}

\begin{figure}[h]
  \captionsetup[subfigure]{justification=centering}
  \centering
  \hypersetup{linkcolor=captionTextColor}
  \includegraphics[width=0.45\linewidth]{sections/design/solar-array/images/ismenius-cavus-solar-panel-starboard-quarter.png}\\
  \caption[Dimensions of starboard quarter panel at Ismenius Cavus]
          {Dimensions of starboard quarter panel at Ismenius Cavus. Measurments are in millimeter. The same dimensions apply for the starboard bow, port bow, and port quarter panels. The length between the base of the panel and the \ac{CoM} is also indicated and corresponds to $r = \SI{385}{\milli\metre}$.}
  \label{fig:ismenius-cavus-solar-panel-starboard-quarter-dimensions}
\end{figure}

\vspace{0.25cm}

The acceleration $a$ required to deploy the port quarter and starboard quarter panels must act against the gravity acceleration $g$. The required acceleration $a$ is set to $g_{Earth} = 9.81 \si{ms^{-2}}$ as a means of introducing margins that account for Martian environment factors such as wind loads. This margin also accounts for imperfections introduced by motor performance degradation factors. Furthermore, sizing motors for an Earth environment reduces design verification and validation costs by eliminating the need for vacuum chamber tests. The panel has a mass of 1.85 \si{\kilo\gram}, thus the required torque $\tau_{torque}$ is:

\begin{align}
  \label{eq:solar-panel-deployment-torque}
  \tau_{torque} &= m \times a \times r\\
                &= 1.85 \times 9.81 \times 0.385\\
                &= 6.99\,\si{\newton\meter}
\end{align}

\clearpage
The \ac{rpm} rotational speed for a \SI{20}{\second} deployment is:

\begin{align}
  \label{eq:solar-panel-deployment-rpm}
  rpm &= \frac{1}{t} \times 60\\
      &= \frac{1}{20} \times 60\\
      &= 3\,{\minute^{-1}}
\end{align}

Transmissions with a 1:30 reduction ratio can be obtained with a planetary gear configuration whereas reduction ratios of 1:100 up to 1:300 are achievable with a harmonic drive. Using a 1:30 transmission will require a motor that must deliver at least 0.23 \si{\newton\meter} of nominal torque and 90 \ac{rpm}. A 1:100 transmission requires a motor with at least $6.99\times\num{e-2}$ \si{\newton\meter} of nominal torque and 300 \ac{rpm}. Finally, a 1:300 transmission requires a motor with at least $2.33\times\num{e-2}$ \si{\newton\meter} of nominal torque and 900 \ac{rpm}.

\subsection{Summary}
\ac{SA} sizing for the worst case power budget of the rover's \textit{Traverse Sol} at optical depth of $\tau = 1$ does not support the use of inclined surfaces for the purpose of increasing traverse time. This is due to solar irradiance being mostly diffuse at high optical depths when airborn dust is responsible for scattered light.

In a dusty atmosphere, an inclined \ac{SA} surface contributes negligeable energy production gains over those obtained with a horizontal configuration. Under certain conditions, these gains are unusable on clear days due to the limited amount of daylight time available for traversing. To resolve this, the \ac{SA} sizing is instead done for the worst case \textit{Hibernation Sol} in which power draws are reduced by introducing \acp{RHU}.

Assumptions, requirements, constraints, and design drivers presented in the previous section are considered in the baseline designs from which deployment sequences and motor requirements are presented. Redesiging the location and stowed posture of the \ac{RA} is a subject for future work towards completely folding the panels over the rover's body. Such a configuration would significantly reduce the stowed rover's vertical span by approximately 1/3.
