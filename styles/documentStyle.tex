% ********************************************************************
% composed by Florian Cordes @ DFKI RIC
%
% to be used with documentclass memoir:
% \documentclass[11pt,a4paper]{memoir}
%
% January 2018
% ********************************************************************

\NeedsTeXFormat{LaTeX2e}
\ProvidesPackage{phdDocumentStyle_FCordes}[2018/01/19 v1.0 Style for PhD document]



%Zeilenumbruch, falls overfull hbox
\sloppy
%\relax

%%%%%%%%%%%%%%%%%%%%%%%%%%%%%%%%%%%%%%%%%%%
%% define what happens when
%% draft option is set in documentclass
%%%%%%%%%%%%%%%%%%%%%%%%%%%%%%%%%%%%%%%%%%%
\newcommand{\myDraftNote}[0]{\color{thesisSecondaryColor}{\textit{Draft version: \today}}}
\newcommand{\myFinalDraftNote}[0]{\color{thesisSecondaryColor}{\textit{Final Draft (\today)}}}

\ifdraftdoc
    \makeevenfoot{plain}{}{\thepage}{\myDraftNote}
    \makeoddfoot{plain}{\myDraftNote}{\thepage}{}
    \makeevenfoot{ruled}{\thepage}{}{\myDraftNote}
    \makeoddfoot{ruled}{\myDraftNote}{}{\thepage}
\fi

\makeevenhead{ruled}{\scshape Chapter \leftmark}{}{}
\makeoddhead{ruled}{}{}{\itshape\rightmark}


%%%%%%%%%%%%%%%%
%% Itimisation
%%%%%%%%%%%%%%%%
\newlength{\sqsize}
\setlength{\sqsize}{0.8ex}
\newcommand*\sq{\raisebox{0.4\sqsize}{\rule{\sqsize}{\sqsize}}} % define a small square for itemize bullet

\renewcommand{\labelitemi}{${\color{itemicolor}\sq}$}%\blacksquare}$}
\renewcommand{\labelitemii}{${\color{itemiicolor}\blacktriangleright}$} %\bullet}$}
\renewcommand{\labelitemiii}{${\color{itemiiicolor}\square}$}




%%%%%%%%%%%%%%%%
%% Description
%%%%%%%%%%%%%%%%
\renewcommand*{\descriptionlabel}[1]{\hspace
                                     \labelsep
                                     \normalfont
                                     \textbf{\color{descriptionColor}{#1}}
                                 }




%%%%%%%%%%%%%%%%%%%
%% footnote formatting and colors
%%%%%%%%%%%%%%%%%%%
%change the mark (reference-number)
\renewcommand\thefootnote{\textcolor{footnoteMarkColor}{
                                            (\alph{footnote})%
                                            %\arabic{footnote})%
                                            }}

%change the text at the bottom of the page
\renewcommand{\foottextfont}{\footnotesize}%\color{footnoteTextColor}}

%change the rule above the footnote
\renewcommand*{\footnoterule}{%
    \kern-3pt%
    \color{footnoteRuleColor}\hrule width 0.4\columnwidth %%also sets the text-body color when written like this
    \kern 2.6pt
    }

\setlength{\footnotesep}{2ex}       % the separation between footnotes
\setlength{\footmarkwidth}{1.5em}   % indention of complete footnote textblock
\setlength{\footmarksep}{0.3em}       % indention of second and following lines
\setlength{\footparindent}{2em}     % indention of paragraph within footnote

%make a refernce-command for footnotes
\makeatletter
\newcommand\fnref[1]{\protected@xdef\@thefnmark{\ref{#1}}\@footnotemark}
\makeatother

%do not reset counter per chapter
\counterwithout*{footnote}{chapter}

% footnotes shall be at the bottom of the page, even wenn a figure has [b] option
\feetbelowfloat

%%%%%%%%%%%%%%%%%%%%%%%%%%%%%%
%% Define sectioning depth
%%%%%%%%%%%%%%%%%%%%%%%%%%%%%%
\setsecnumdepth{subsection}
\settocdepth{section}


%%%%%%%%%%%%%%%%%%%%%%%%%%%%%%
%% new line spacing
%%%%%%%%%%%%%%%%%%%%%%%%%%%%%%
\setSingleSpace{1.05}
\SingleSpacing





%%%%%%%%%%%%%%%%%%%%%%%%%%%%%%%%%%%%%%%%%%%
%% define the text field for the pages
%%%%%%%%%%%%%%%%%%%%%%%%%%%%%%%%%%%%%%%%%%%
\newlength{\myRegularParskip}
\setlength{\myRegularParskip}{1.8mm}

\newlength{\myCenterSpaceOffset}
\setlength{\myCenterSpaceOffset}{0.7mm} % pos increases spacing in center of double page

\textheight235mm              % Höhe des Textbereichs
\textwidth160mm             % Breite des Textbereichs
\topmargin-20mm
\topskip5mm
\headheight20mm
\headsep3mm
\setlength\evensidemargin{-\myCenterSpaceOffset} % pos vals to move in
\setlength\oddsidemargin{\myCenterSpaceOffset}   % neg vals to move in
\footskip15mm
\headwidth\textwidth

\setlength\parindent{0mm}
\setlength\parskip{\myRegularParskip}


%%%%%%%%%%%%%%%%%%%%%%%%%%%%%%
%% Defines for Chapter Style
%%%%%%%%%%%%%%%%%%%%%%%%%%%%%%
\usepackage{kpfonts}

\newcommand\numlifter[1]{\raisebox{-2.5cm}[0pt][0pt]{\smash{#1}}} % how low can you go?
\newcommand\numindent{\kern10pt} %space from right margin
\newlength\chaptertitleboxheight
\makechapterstyle{cordesDiss}{
  \renewcommand\printchaptername{\raggedleft}
  \renewcommand\printchapternum{%
    \begingroup%
    \leavevmode%
    \chapnumfont%
    \strut%
    \numlifter{\thechapter}%{\fontfamily{pnc}\selectfont \thechapter} }%
    \numindent%
\endgroup%
}
  \renewcommand*{\printchapternonum}{%
    \vphantom{\begingroup%
      \leavevmode%
      \chapnumfont%
      \numlifter{\vphantom{9}}%
      \numindent%
      \endgroup}
    \afterchapternum}
  \setlength\midchapskip{0pt}
  \setlength\beforechapskip{0.5\baselineskip}
  \setlength{\afterchapskip}{3\baselineskip}
  \renewcommand\chapnumfont{%
    \fontsize{4cm}{0cm}%
    \bfseries%
    \sffamily%
    \color{chaptercolor}%
  }
  \renewcommand\chaptitlefont{%
    \normalfont%
    \huge%
    \bfseries%
    \raggedleft%
  }%
  \settototalheight\chaptertitleboxheight{%
    \parbox{\textwidth}{\chaptitlefont \strut bg\\bg\strut}}
  \renewcommand\printchaptertitle[1]{%
    \parbox[t][\chaptertitleboxheight][t]{\textwidth}{%
      %\microtypesetup{protrusion=false}% add this if you use microtype
      \chaptitlefont\fontsize{0.9cm}{\baselineskip}\selectfont\strut ##1\strut}%FC
      %\chaptitlefont \strut ##1\strut}% ORIGINAL
      %\HUGE\bfseries\strut ##1\strut}% FC
}}

%% now set the style active
\chapterstyle{cordesDiss}
\aliaspagestyle{chapter}{plain} % just to save some space: no header

%\createplainmark{toc}{both}{\acronym}


%% -----------------------------
%% floating environment options
%% -----------------------------

% define how much text shall be on same page with large images
%% standard seems to be 0.20 --> 20%
%\renewcommand{\textfraction}{0.01}


%% setup caption style
\captionsetup[figure]{labelfont={bf,color=captionCatergoryColor},textfont={it,color=captionTextColor},justification=raggedright,singlelinecheck=false,format=hang}
\captionsetup[table]{labelfont={bf,color=captionCatergoryColor},textfont={it,color=captionTextColor},justification=raggedright,singlelinecheck=false,format=hang}
%\captionsetup[table]{skip=7pt} %% mehr platz zwischen tabelle und caption

%% Durchgängige Nummerierung von Figures und Tables:
%\usepackage{chngcntr}
%\counterwithout{figure}{chapter}
%\counterwithout{table}{chapter}



%% --------------------
%% special functions
%% --------------------

% acronym
\newcommand{\dissAcroextra}[1]{\acroextra{~\\[-0.6ex]\textcolor{acroextraColor}{\scriptsize{(#1)}}}}

% quote at beginning of chapter
\newcommand{\shinyChapterQuote}[2]{
\begin{quote}\hypersetup{hidelinks=true}
    \textcolor{quoteColor}{
        \emph{#1}
        \newline\indent\qquad -- #2
    }
\end{quote}
}

% a command to add (smaller) extra info in captions of Figures and Tables
\newcommand{\dissExtraCaption}[1]{\newline\footnotesize{#1}}

%% symbols
\newcommand{\dissCheck}{\textcolor{thesisMainColor}{$\mathbf{\checkmark}$}}
\newcommand{\dissUncheck}{\textcolor{thesisSecondaryColor}{$\mathbf{\times}$}}
\newcommand{\myTTAmark}{\textcolor{thesisSecondaryColor!80}{\ensuremath{^\star}}\xspace}%{\ensuremath{^\circledast}}\xspace}%{\ensuremath{^\blacklozenge}}\xspace}
